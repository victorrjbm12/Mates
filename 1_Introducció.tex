\chapter{Introducci\'o}
L'estudi de la geometria és un dels més antics de la història de les matemàtiques. El mot \textit{geometria}, que significa literalment \say{mesura de la Terra}, indica el lligam de la geometria amb la seva utilitat en la distribució de terres de conreus i en la construcció arquitectònica. Les primeres aplicacions de la geometria es troben en el desenvolupament a Egipte i Babilònia de tècniques de càlcul de distàncies, àrees i volums. Les primeres escoles filosòfiques de l'Antiga Grècia es van interessar profundament en la geometria, emprenent un llarg camí que portaria a una abstracció cada vegada més gran de les matemàtiques respecte les seves aplicacions pràctiques, invertint l'ordre ontològic de tal manera que les matemàtiques fossin el fonament del món, amb Pitàgores i Plató.

El primer en escriure un tractat de geometria en el sentit modern, amb la demostració de teoremes i proposicions partint d'axiomes i definicions va ser Euclides, en \textit{Els Elements}. El famós cinquè postulat dels Elements, el \textit{postulat de les paral·leles}, implica que les rectes paral·leles no es tallen en cap punt. Aquesta afirmació, que sembla molt natural, resulta no ser certa en certs espais geomètrics. Un d'ells és molt quotidià: la superfície terrestre. Com va demostrar el també grec Eratòstenes, la Terra és esfèrica, de manera que qualsevol parell de rectes aparentment paral·leles sobre la superfície d'aquesta han de trobar-se en algun punt. Així, obtenim el primer tipus de geometria no euclidiana: la geometria esfèrica. En el cas de la Terra obtenim una distinció que acabaria sent molt important entre dues maneres de concebre la geometria: el punt de vista intrínsec i el punt de vista extrínsec. El punt de vista extrínsec estudiaria la geometria de l'esfera com una regió concreta de l'espai euclidià tridimensional, de manera que cap corba sobre la superfície de la Terra traça res que es pugui anomenar una recta. Les úniques rectes vertaderes són tangents a la superfície terrestre. El punt de vista intrínsec, en canvi, estudia la geometria esfèrica en si mateixa, sense referència a l'espai euclidia en què es troba. En aquesta nova geometria, el cinquè postulat d'Euclides i les proposicions que se'n deriven no són vàlides.

La branca de la geometria que ens interessa en aquest treball, la geometria diferencial, no va néixer fins segles més tard. Va ser necessaria passar per la formalització de l'àlgebra i la geometria mitjançant coordenades de la mà de Descartes, i el desenvolupament del càlcul infinitesimal de Newton i Leibniz, per poder desenvolupar la geometria diferencial. L'ús de coordenades permet treballar en espais més abstractes i de dimensió arbitrària, mentre el càlcul infinitesimal permet estudiar les propietats de les funcions que els defineixen. Després d'aquesta formalització és quan apareixen alguns dels matemàtics que més aportarien a la geometria diferencial, i els noms dels quals apareixeran diverses vegades al llarg del treball. En concret, parlem de Carl Friedrich Gauss i Bernhard Riemann.

Gauss, estudiant les propietats de les superfícies i les corbes en l'espai euclidià tridimensional, enuncia un dels teoremes més importants de la geometria diferencial: el \textbf{teorema egregium} {\color{blue} alguna nota sobre el significat de egregi}. Segons aquest teorema, la curvatura gaussiana d'una superfície regular en l'espai euclidià tridimensional és invariant per isometries, i es pot determinar a partir de les propietats intrínseques de la superfície. Aquest resultat explica per què no es pot el·laborar un mapa pla de la superfície terrestre que conservi distàncies i angles.
Riemann solidifica el gir intrínsec en la geometria diferencial, iniciant l'estudi de varietats diferenciables de dimensió superior i amb mètriques arbitràries. Riemann generalitza el concepte de curvatura a través de l'àlgebra multilineal i desenvolupa les eines per a l'estudi de qualsevol varietat diferenciable amb mètrica. Poc després, amb la introducció de les mètriques pseudo-riemannianes, la geometria diferencial seria una eina essencial pel desenvolupament de la teoria de la relativitat general, que és encara la millor teoria física per descriure la relació entre la massa i la curvatura de l'espai-temps.

Tot i els avenços en l'estudi intrínsec de les varietats diferenciables, encara hi ha molt a dir sobre la manera en què aquestes varietats es poden encabir (en anglès, \textit{embed}) o immergir en espais euclidians de dimensió superior. Com veurem al capítol \ref{cap:intro}, el teorema de Whitney afirma que tota varietat diferenciable suau $n$-dimensional es pot encabir de manera suau en un espai euclidià de dimensió $2n$. Ara bé, aquest encabiment no és necessàriament isomètric, de manera que la geometria d'aquesta imatge no és necessàriament la geometria intrínseca de la varietat, en el sentit riemannià. Un dels exemples més coneguts d'aquesta limitació és el de la immersió de superfícies regulars compactes en $\mathbb R^3$.


\subsection*{Objectius del treball}



\begin{itemize}
    \item Explicats
\end{itemize}

\subsection*{Estructura de la mem\`oria}
Tremendo
HEM DE MIRAR DE POSAR LA DEMOSTRACIÓ DE QUE NOSEQUÈ SEMPRE HI HA UN PUNT EL·LIPTIC
\subsection*{Guia de lectura}
Faig servir encabiment, que potser hauria de dir immersió?

\pagenumbering{arabic}
\setcounter{page}{1}


\section{Temes de les reunions}
Sigui $\mathcal M$ una varietat diferenciable amb una distància $d$, i sigui $$\nu = \set{(x,v)\in \mathcal M \times \mathbb R^n: v\in T_x\mathcal M^{\perp}}.$$ Per qualsevol $\varepsilon > 0$, definim el subconjunt $$\nu_\varepsilon = \set{(x,v)\in \nu: ||v|| < \varepsilon}$$
i l'aplicació $$\begin{aligned}
    \sigma:\nu_\varepsilon &\to \mathbb R^n \\
    (x,v) &\mapsto x+v
\end{aligned}$$
\begin{teo}
    Si $\varepsilon$ és prou petit, aleshores $\sigma:\nu_\varepsilon \to \sigma(\nu_\varepsilon)$ és homeomorfisme.
\end{teo}
{\color{green!50!black}\textit{Prova. (Meva, està molt millor al John M. Lee)} Primer, volem veure que $\sigma$ és injectiva. Suposem que $\mathcal M$ és $C^\infty$ (amb $C^2$ hauria de ser prou). Donat qualsevol punt $x\in\mathcal M$ existeix un entorn prou petit $U_x$ de $x$ en $\mathcal M$ tal que $\nu_1$ és injectiva. Això és degut al fet que, localment, la varietat és aproximadament igual al seu espai tangent. 

Sigui $x_0$ el punt amb l'entorn $U_{x_0}$ més petit que verifica la propietat anterior, i sigui $y_0$ el punt de $\mathcal M\setminus U_{x_0}$ tal amb el vector $w_0\in T_{y_0}\mathcal M$ més curt tal que existeix algun $v_0\in T_{x_0}\mathcal M$ tal que $x_0+v_0=y_0+w_0$. 
Sigui $l=||w_0||$. Aleshores, posant $\varepsilon = l/2$, tenim que $\sigma$ és injectiva. 

Pel que fa a la exhaustivitat, $\sigma:\nu_\varepsilon \to \sigma(\nu_\varepsilon)$ és exhaustiva per definició. A més, en ser la suma de dos vectors en $\mathbb R^n$, $\sigma$ és contínua.

Cal veure que la inversa $\sigma^{-1}$ és contínua. Sigui $a\in \sigma(\nu_\varepsilon)$. Aleshores, existeixen $x\in \mathcal M$ i $v\in \mathbb R^n$ tals que $a=\sigma(x,v)=x+v$. Com $\sigma^{-1}$ projecta punts de $\sigma(\nu_\varepsilon)$ en el punt de $\nu_\varepsilon$ més proper, tenim que $\sigma^{-1}$ és contínua.
Per tant, $\sigma$ és un homeomorfisme. \qed}
\begin{obs} 
    D'aquí surt el que fa servir Nash per allò del conjunt que no admet dues perpendiculars.
\end{obs}
\newpage