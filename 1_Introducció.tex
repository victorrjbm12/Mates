\chapter{Introducci\'o}



\subsection*{Objectius del treball}

\begin{itemize}
    \item Explicats
\end{itemize}

\subsection*{Estructura de la mem\`oria}
Tremendo
HEM DE MIRAR DE POSAR LA DEMOSTRACIÓ DE QUE NOSEQUÈ SEMPRE HI HA UN PUNT EL·LIPTIC
\subsection*{Guia de lectura}
Faig servir incrustació, que potser hauria de dir immersió?

\newpage

\pagenumbering{arabic}
\setcounter{page}{1}


\chapter{Comencem}
\section{Temes de les reunions}
Sigui $\mathcal M$ una varietat diferenciable amb una distància $d$, i sigui $$\nu = \set{(x,v)\in \mathcal M \times \mathbb R^n: v\in T_x\mathcal M^{\perp}}.$$ Per qualsevol $\varepsilon > 0$, definim el subconjunt $$\nu_\varepsilon = \set{(x,v)\in \nu: ||v|| < \varepsilon}$$
i l'aplicació $$\begin{aligned}
    \sigma:\nu_\varepsilon &\to \mathbb R^n \\
    (x,v) &\mapsto x+v
\end{aligned}$$
\begin{teo}
    Si $\varepsilon$ és prou petit, aleshores $\sigma:\nu_\varepsilon \to \sigma(\nu_\varepsilon)$ és homeomorfisme.
\end{teo}
{\color{green!50!black}\textit{Prova. (Meva, està molt millor al John M. Lee)} Primer, volem veure que $\sigma$ és injectiva. Suposem que $\mathcal M$ és $C^\infty$ (amb $C^2$ hauria de ser prou). Donat qualsevol punt $x\in\mathcal M$ existeix un entorn prou petit $U_x$ de $x$ en $\mathcal M$ tal que $\nu_1$ és injectiva. Això és degut al fet que, localment, la varietat és aproximadament igual al seu espai tangent. 

Sigui $x_0$ el punt amb l'entorn $U_{x_0}$ més petit que verifica la propietat anterior, i sigui $y_0$ el punt de $\mathcal M\setminus U_{x_0}$ tal amb el vector $w_0\in T_{y_0}\mathcal M$ més curt tal que existeix algun $v_0\in T_{x_0}\mathcal M$ tal que $x_0+v_0=y_0+w_0$. 
Sigui $l=||w_0||$. Aleshores, posant $\varepsilon = l/2$, tenim que $\sigma$ és injectiva. 

Pel que fa a la exhaustivitat, $\sigma:\nu_\varepsilon \to \sigma(\nu_\varepsilon)$ és exhaustiva per definició. A més, en ser la suma de dos vectors en $\mathbb R^n$, $\sigma$ és contínua.

Cal veure que la inversa $\sigma^{-1}$ és contínua. Sigui $a\in \sigma(\nu_\varepsilon)$. Aleshores, existeixen $x\in \mathcal M$ i $v\in \mathbb R^n$ tals que $a=\sigma(x,v)=x+v$. Com $\sigma^{-1}$ projecta punts de $\sigma(\nu_\varepsilon)$ en el punt de $\nu_\varepsilon$ més proper, tenim que $\sigma^{-1}$ és contínua.
Per tant, $\sigma$ és un homeomorfisme. \qed}
\begin{obs} 
    D'aquí surt el que fa servir Nash per allò del conjunt que no admet dues perpendiculars.
\end{obs}
\newpage