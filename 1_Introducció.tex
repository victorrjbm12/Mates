\chapter{Introducci\'o}

La geometria diferencial és l'estudi de les varietats diferenciables, és a dir, de varietats topològiques amb estructures diferenciables que permeten realitzar-hi càlcul infinitesimal. Si aquestes varietats diferenciables són subconjunts d'espais mètrics, es poden estudiar les seves propietats geomètriques induïdes per la mètrica de l'espai ambient en què es troben, el que anomenem \textit{geometria extrínseca}. Ara bé, les varietats diferenciables es poden dotar de mètriques riemannianes intrínseques, que permeten mesurar distàncies i angles locals sense fer referència a cap espai ambient. Anomenem les varietats diferenciables amb mètriques riemannianes \textit{varietats riemannianes}.

Aquests dos punts de vista envers les varietats diferenciables, com espais en si mateixos i com subconjunts d'altres espais, donen lloc a moltes preguntes pel que fa a la seva relació. Una d'elles és si tota varietat diferenciable es pot encabir (en anglès, \textit{embed}) o immergir (en anglès, \textit{immerse}) en un espai euclidià de dimensió superior. Com veurem al capítol \ref{cap:intro}, el teorema de Whitney afirma que tota varietat diferenciable $n$-dimensional suau, és a dir, de classe $C^\infty$, es pot encabir amb regularitat $C^\infty$ en un espai euclidià de dimensió $2n$. Ara bé, si aquesta varietat diferenciable és riemanniana, el teorema de Whitney no ens assegura que la immersió o l'encabiment sigui isomètric. És a dir, la mètrica induïda per l'espai ambient sobre la imatge de la varietat riemanniana per aquesta aplicació no coincidirà, en general, amb la mètrica intrínseca de la varietat. 

Un dels resultats clàssics pel que fa a la relació entre les mètriques intrínseca i extrínseca és el teorema egregi de Gauss, que veurem al capítol \ref{cap:intro}, segons el qual la curvatura gaussiana d'una superfície regular en l'espai euclidià tridimensional és invariant per isometries i es pot determinar a partir de les propietats intrínseques de la superfície. En efecte, aquest teorema troba certes restriccions a l'hora d'encabir o immergir de manera isomètrica i $C^\infty$ una varietat riemanniana en $\mathbb R^3$. Un dels exemples més importants d'aquestes limitacions és el fet que qualsevol superfície regular compacta immergida en $\mathbb R^3$ ha de tenir algun punt el·líptic. Això implica, com veurem en aquest treball, que cap tor pla es pot encabir de manera isomètrica i $C^\infty$ en $\mathbb R^3$.

Janet i \citet{cartan1927} van obtenir alguns resultats primerencs en l'estudi dels encabiments isomètrics de classe $C^\infty$ en espais euclidians arbitraris, demostrant que en el cas de varietats analítiques es pot trobar un encabiment suau localment isomètric en espais de dimensió majors o iguals a la dimensió de Janet, $s_n = \frac{n(n+1)}{2}$. Aquest resultat es va estendre posteriorment per part de Gromov i \citet{rokhlin1970} a espais de dimensió $q\ge s_n+n$ en el cas de varietats $C^\infty$. Com veiem, els espais ambients que asseguren l'existència d'encabiments isomètrics suaus poden arribar a ser de dimensió molt més alta que els que apareixen al teorema de Whitney.

La impossibilitat de trobar un encabiment isomètric suau del tor pla en $\mathbb R^3$ és deguda, en última instància, al fet que el tor pla té curvatura gaussiana nul·la, de manera que no pot tenir cap punt el·líptic. Ara bé, la curvatura d'una superfície regular està íntimament relacionada amb les segones derivades de les corbes que es poden definir sobre la varietat riemanniana, de manera que no està ben definida sobre la imatge de l'encabiment o immersió si aquesta aplicació no és $C^2$. És possible trobar un encabiment isomètric del tor pla si relaxem la condició de regularitat a $C^1$?

L'any 1954, John Forbes Nash Jr. va publicar un article sorprenent en què demostrava que per qualsevol encabiment $C^\infty$ d'una varietat riemanniana estrictament curt, és a dir, tal que escurci localment les distàncies entre punts, es pot trobar un encabiment isomètric de classe $C^1$ arbitràriament proper, sempre que la codimensió de l'espai ambient sigui major o igual que 2. Per fer-ho, va dissenyar un mètode iteratiu en cada etapa del qual la imatge de l'encabiment és modificada amb una mena de moviment espiral. Després d'un nombre finit d'etapes, s'obté un nou encabiment curt $C^\infty$, però el procés convergeix en un encabiment isomètric de classe $C^1$, pel qual la curvatura extrínseca de la seva imatge no està ben definida. La demostració és constructiva, però el requeriment de dues direccions normals a la imatge de l'encabiment impossibilita la seva aplicació per superfícies encabides en l'espai euclidià tridimensional. Un any més tard, Nicolaas H. Kuiper va trobar un mètode similar al de Nash que rebaixa la codimensió de l'espai ambient a $1$. La seva modificació del mètode de Nash substitueix el moviment espiral per una corrugació que només requereix l'existència d'una direcció normal a la imatge de l'encabiment.

Amb aquest rerefons, en aquesta memòria es treballarà des dels seus fonaments el problema dels encabiments isomètrics $C^1$ i $C^\infty$ de varietats riemannianes en espais euclidians, parant especial atenció a dos casos concrets de superfícies encabides en $\mathbb R^3$ per a les quals hi ha hagut desenvolupaments recents. En concret, estudiarem les cintes de Möbius planes rectangulars i els tors plans. Les cintes de Möbius planes rectangulars són aquelles obtingudes identificant dos costats d'un rectangle euclidià. Anomenem \textit{raó d'aspecte} la raó entre la longitud del costat més llarg i el costat més curt d'aquesta cinta. Demostrarem que és necessària una raó d'aspecte $\lambda>\sqrt3$ per tal que existeixi un encabiment isomètric suau en $\mathbb R^3$. 
Veurem també que, si es relaxa la condició de regularitat, és possible trobar un encabiment isomètric per a qualsevol cinta de Möbius en $\mathbb R^3$. En concret, es pot construir una aplicació contínua que sigui un encabiment $C^\infty$ a trossos. Pel que fa als tors plans, veurem la demostració que no existeix cap encabiment isomètric $C^\infty$ d'aquests en $\mathbb R^3$. Després de demostrar el teorema de Nash d'encabiments isomètrics $C^1$, veurem una aplicació de les tècniques que hi desenvolupa per trobar un encabiment isomètric d'un tor pla en $\mathbb R^3$.

\subsection*{Objectius del treball}
\begin{itemize}
    \item Donar una introducció a la geometria diferencial i a la geometria Riemanniana, amb especial atenció als encabiments isomètrics i a la regularitat de les aplicacions que els defineixen.
    \item Demostrar la impossibilitat d'encabir isomètricament el tor pla en $\mathbb R^3$ de manera suau.
    \item Demostrar el resultat recent de \citet{schwartz2024} que troba una raó d'aspecte mínima per encabir una cinta de Möbius isomètricament en $\mathbb R^3$ amb regularitat $C^\infty$, i mostrar que no existeix aquesta raó d'aspecte mínima si es relaxa la condició de regularitat de manera adequada.
    \item Enunciar i demostrar els teoremes de Nash d'immersions i encabiments $C^1$, i explicar el refinament de Kuiper.
    \item Mostrar el mètode de \citet{borrelli2013} per encabir de manera isomètrica $C^1$ el tor pla en $\mathbb R^3$, directament relacionat amb el teorema de Nash-Kuiper. 
\end{itemize}

\subsection*{Estructura de la mem\`oria}
El treball es divideix en cinc capítols, relacionats amb els objectius exposats a l'apartat anterior. El primer d'ells és aquesta mateixa introducció. El capítol 2 comença amb un breu recordatori de la definició de les classes de regularitat de funcions, per després centrar-se en la geometria diferencial i Riemanniana. Hi definim els conceptes necessaris per a la resta del treball, enunciant i demostrant els resultats més importants. També hi enunciem alguns teoremes importants, com el teorema de Sard i el teorema egregi de Gauss, sense donar-ne la demostració. Al final del capítol discutim la impossibilitat d'encabir isomètricament el tor pla en $\mathbb R^3$. El capítol 3 segueix la demostració completa d'un dels dos teoremes demostrats recentment sobre la cinta de Möbius \cite{schwartz2024}, enunciant també l'altre resultat. A més, expliquem una aplicació contínua que és un encabiment $C^\infty$ a trossos per la qual no hi ha raó d'aspecte mínima. El capítol 4 segueix l'article original de Nash en què demostra els seus teoremes sobre encabiments i immersions $C^1$. Algunes de les demostracions d'aquest capítol s'han reestructurat i fet més explícites que a l'article original. Acabem aquest capítol explicant el punt clau del refinament de Kuiper. Finalment, al capítol 5 exposem el treball de \citet{borrelli2013} per trobar un encabiment isomètric $C^1$ del tor pla en $\mathbb R^3$, sense donar massa detalls en les demostracions. 
\subsection*{Glossari de termes traduits al català}
Totes les fonts utilitzades en la redacció d'aquesta memòria són en anglès, de manera que ha calgut traduir alguns termes tècnics. Si bé la majoria d'aquests tenen traducció estàndard al català, alguns d'ells no en tenen. El cas més important per aquest treball és el del terme \textit{embedding}, que sovint s'escriu en anglès i en cursiva, però aquí apareixerà com \textit{encabiment}. A continuació introduïm un petit glossari d'alguns termes anglesos i la traducció per la qual hem optat en aquest treball. 

\begin{center}
    \begin{tabular}{ll}
    \hline
    \textbf{Terme anglès} & \textbf{Traducció al català} \\
    \hline
    \textit{Aspect ratio} & \textit{Raó d'aspecte} \\
    \textit{Bend} & \textit{Plec} \\
    \textit{Bump function} & \textit{Funció de relleu} \\
    \textit{Embedding} & \textit{Encabiment} \\
    \textit{Exhaustion function} & \textit{Funció d'exhauriment} \\
    \textit{Flatness} & \textit{Planor} \\
    \textit{Isometric default} & \textit{Defecte isomètric} \\
    \textit{Loops} & \textit{Voltes} \\
    \textit{Pre-bend} & \textit{Pre-plec} \\
    \textit{Smooth} & \textit{Suau} \\
    \textit{Stage} & \textit{Etapa} \\
    % \textit{Step} & \textit{Pas} \\
    \textit{Strain} & \textit{Corrugació} \\
    \hline
    \end{tabular}
\end{center}
\newpage
\pagenumbering{arabic}
\setcounter{page}{1}
