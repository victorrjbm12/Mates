\chapter{Introducci\'o}

La geometria diferencial és l'estudi de les varietats diferenciables, és a dir, d'espais geomètrics localment euclidians amb estructures diferenciables que permeten realitzar-hi càlcul infinitesimal. Si aquestes varietats diferenciables són subconjunts d'espais mètrics, es poden estudiar les seves propietats geomètriques induïdes per la mètrica de l'espai ambient en què es troben, el que anomenem la seva geometria extrínseca. Ara bé, les varietats diferenciables es poden dotar de mètriques riemannianes intrínseques, que permeten mesurar distàncies i angles locals sense fer referència a cap espai ambient. Anomenem les varietats diferenciables amb mètriques riemannianes \textit{varietats riemannianes}.

Aquests dos punts de vista envers les varietats diferenciables, com espais en si mateixos i com subconjunts d'altres espais, dona lloc a moltes preguntes pel que fa a la seva relació. Una d'elles és si tota varietat diferenciable es pot encabir (en anglès, \textit{embed}) o immergir (en anglès, \textit{immerse}) en un espai euclidià de dimensió superior. Com veurem al capítol \ref{cap:intro}, el teorema de Whitney afirma que tota varietat diferenciable $n$-dimensional suau, és a dir, de classe $C^\infty$, es pot encabir amb regularitat $C^\infty$ en un espai euclidià de dimensió $2n$. Ara bé, si aquesta varietat diferenciable és riemanniana, el teorema de Whitney no ens assegura que la immersió o l'encabiment sigui isomètric. És a dir, la mètrica induïda per l'espai ambient sobre la imatge de la varietat riemanniana per aquesta aplicació no serà equivalent a la mètrica intrínseca de la varietat. 

Un dels resultats clàssics pel que fa a la relació entre les mètriques intrínseca i extrínseca és el teorema egregi de Gauss, que també veurem al capítol \ref{cap:intro}, segons el qual la curvatura gaussiana d'una superfície regular en l'espai euclidià tridimensional és invariant per isometries, i es pot determinar a partir de les propietats intrínseques de la superfície. En efecte, aquest teorema troba certes restriccions a l'hora d'encabir o immergir de manera isomètrica i $C^\infty$ una varietat riemanniana en $\mathbb R^3$. Un dels exemples més importants d'aquestes limitacions és el fet que qualsevol superfície regular compacta immergida en $\mathbb R^3$ ha de tenir algun punt el·líptic. Això implica, com veurem en aquest treball, que no existeix cap encabiment isomètric suau del tor pla en $\mathbb R^3$.

Alguns resultats primerencs en l'estudi dels encabiments isomètrics de classe $C^\infty$ en espais euclidians arbitraris van ser obtinguts per Janet i Cartan, demostrant que en el cas de varietats analítiques es pot trobar un encabiment suau localment isomètric en espais de dimensió majors o iguals a la dimensió de Janet, $s_n = \frac{n(n+1)}{2}$. Aquest resultat es va expandir posteriorment per part de Gromov i Rokhlin a espais de dimensió $q\ge s_n+n$ en el cas de varietats $C^\infty$. Com veiem, els espais ambients necessaris per trobar encabiments isomètrics suaus poden arribar a ser de dimensió molt més alta que els que apareixen al teorema de Whitney.









% L'estudi de la geometria és un dels més antics de la història de les matemàtiques. El mot \textit{geometria}, que significa literalment \say{mesura de la Terra}, denota la seva utilitat en la distribució de terres de conreu i en la construcció arquitectònica. Les primeres aplicacions de la geometria es troben en el desenvolupament a Egipte i Babilònia de tècniques de càlcul de distàncies, àrees i volums. Les primeres escoles filosòfiques de l'Antiga Grècia es van interessar profundament en la geometria, emprenent un llarg camí que portaria a una abstracció cada vegada més gran de les matemàtiques respecte les seves aplicacions pràctiques, invertint l'ordre ontològic de tal manera que les matemàtiques fossin el fonament del món, amb Pitàgores i Plató.

% El primer en escriure un tractat de geometria en el sentit modern, amb la demostració de teoremes i proposicions partint d'axiomes i definicions va ser Euclides, en \textit{Els Elements}. El famós cinquè postulat dels Elements, el \textit{postulat de les paral·leles}, implica que les rectes paral·leles no es tallen en cap punt. Aquesta afirmació, que sembla molt natural, resulta no ser certa en certs espais geomètrics. Un d'ells és molt quotidià: la superfície terrestre. Com va demostrar el també grec Eratòstenes, la Terra és esfèrica, de manera que qualsevol parell de rectes aparentment paral·leles sobre la superfície d'aquesta han de trobar-se en algun punt. Així, obtenim el primer tipus de geometria no euclidiana: la geometria esfèrica. En el cas de la Terra, obtenim una distinció que acabaria sent molt important entre dues maneres de concebre la geometria: el punt de vista intrínsec i el punt de vista extrínsec. El punt de vista extrínsec estudiaria la geometria de l'esfera com una regió concreta de l'espai euclidià tridimensional, de manera que cap corba sobre la superfície de la Terra traça res que puguem anomenar una recta. Les úniques rectes vertaderes són tangents a la superfície terrestre. El punt de vista intrínsec, en canvi, estudia la geometria esfèrica en si mateixa, sense referència a l'espai euclidià en què es troba. En aquesta nova geometria, el cinquè postulat d'Euclides i les proposicions que se'n deriven no són vàlids.

% La branca de la geometria que ens interessa en aquest treball, la geometria diferencial, no va néixer fins segles més tard. Per poder desenvolupar-la, va ser necessària la formalització de l'àlgebra i la geometria mitjançant coordenades de la mà de Descartes, i el desenvolupament del càlcul infinitesimal de Newton i Leibniz, per poder desenvolupar la geometria diferencial. L'ús de coordenades permet treballar en espais més abstractes i de dimensió arbitrària, mentre el càlcul infinitesimal permet estudiar les propietats de les funcions que els defineixen. Després d'aquesta formalització és quan apareixen alguns dels matemàtics que més aportarien a la geometria diferencial, i els noms dels quals apareixeran diverses vegades al llarg del treball. En concret, parlem de Carl Friedrich Gauss i Bernhard Riemann.

% Gauss, estudiant les propietats de les superfícies i les corbes en l'espai euclidià tridimensional, enuncia un dels teoremes més importants de la geometria diferencial: el \textbf{teorema egregium}. Segons aquest teorema, la curvatura gaussiana d'una superfície regular en l'espai euclidià tridimensional és invariant per isometries, i es pot determinar a partir de les propietats intrínseques de la superfície. Aquest resultat explica per què no es pot el·laborar un mapa pla de la superfície terrestre que conservi distàncies i angles.
% Riemann solidifica el gir intrínsec en la geometria diferencial, iniciant l'estudi de varietats diferenciables de dimensió superior i amb mètriques arbitràries. Riemann generalitza el concepte de curvatura a través de l'àlgebra multilineal i desenvolupa les eines per a l'estudi de qualsevol varietat diferenciable amb mètrica. Poc després, amb la introducció de les mètriques pseudo-riemannianes, la geometria diferencial seria una eina essencial pel desenvolupament de la teoria de la relativitat general, que és encara la millor teoria física per descriure la relació entre la massa i la curvatura de l'espai-temps.





La impossibilitat de trobar un encabiment isomètric suau del tor pla en $\mathbb R^3$ és deguda, en última instància, al fet que la curvatura del tor pla és incompatible amb la curvatura que ha de tenir per tal de ser encabit d'aquesta manera en $\mathbb R^3$. Ara bé, és possible trobar un encabiment isomètric que no sigui suau? La curvatura d'una superfície regular està íntimament relacionada amb les segones derivades de les corbes que es poden definir sobre la varietat riemanniana, de manera que no està ben definida sobre la imatge de l'encabiment o immersió si aquesta aplicació no és $C^2$. Aquesta és la intuïció que ens porta a estudiar el problema dels encabiments isomètrics en el cas de varietats $C^1$.

L'any 1954, el futur guanyador del Premi Nobel d'Economia John Forbes Nash Jr. va publicar un article sorprenent en què demostrava que es pot prendre qualsevol encabiment de regularitat $C^\infty$ d'una varietat riemanniana estrictament curt, és a dir, tal que escurci localment les distàncies entre punts, i modificar-lo per tal d'obtenir un encabiment isomètric de classe $C^1$, sempre que la codimensió de l'espai ambient sigui major o igual que 2. Per fer-ho, dissenya un mètode iteratiu en cada etapa del qual la imatge de l'encabiment és modificada amb una mena de moviment espiral. Després d'un nombre finit d'etapes, s'obté un nou encabiment curt $C^\infty$, però el procés convergeix en un encabiment isomètric de classe $C^1$, pel qual la curvatura de la varietat encabida no està ben definida. La demostració és constructiva, però el requeriment que la codimensió sigui $2$ impossibilitava la seva aplicació per superfícies encabides en l'espai euclidià tridimensional. Un any més tard, Nicolaas H. Kuiper va demostrar la conjectura de Nash segons la qual podria existir un mètode similar que rebaixés la codimensió de l'espai ambient a $1$. La seva modificació del mètode de Nash substitueix el moviment espiral per una corrugació que només requereix l'existència d'una direcció normal a la imatge de l'encabiment.

Amb aquest rerefons, en aquesta memòria es treballarà des dels seus fonaments el problema dels encabiments isomètrics de varietats riemannianes en espais euclidians, parant especial atenció a dos casos concrets de superfícies encabides en $\mathbb R^3$ pels quals hi ha hagut desenvolupaments recents. En concret, estudiarem la cinta de Möbius i el tor pla. Per la primera varietat, demostrarem que és necessària una raó d'aspecte $\lambda>\sqrt3$ per tal que existeixi un encabiment isomètric suau, i per la segona demostrarem que no existeix cap encabiment isomètric $C^\infty$. Veurem també que podrem trobar un encabiment isomètric per qualsevol cinta de Möbius en $\mathbb R^3$ relaxant regularitat d'aquest a $C^\infty$ a trossos, de manera que hi hagi un subconjunt de punts amb regularitat $C^0$. Demostrarem el teorema de Nash d'encabiments i immersions $C^1$, explicant també la corrugació de Kuiper, i veurem un encabiment explícit del tor pla en $\mathbb R^3$ de regularitat $C^1$.

\subsection*{Objectius del treball}
\begin{itemize}
    \item Donar una introducció a la geometria diferencial i a la geometria Riemanniana, amb especial atenció als encabiments isomètrics i a la regularitat de les aplicacions que els defineixen.
    \item Demostrar la impossibilitat d'encabir isomètricament el tor pla en $\mathbb R^3$ de manera suau.
    \item Demostrar el resultat recent de \cite{schwartz2024} que troba una raó d'aspecte mínima per encabir una cinta de Möbius isomètricament en $\mathbb R^3$ amb regularitat $C^\infty$, i mostrar que no existeix aquesta raó d'aspecte mínima si es relaxa la condició de regularitat de manera adequada.
    \item Enunciar i demostrar els teoremes de Nash d'immersions i encabiments $C^1$, i explicar el refinament de Kuiper.
    \item Mostrar el mètode de \cite{borrelli2013} per encabir de manera isomètrica $C^1$ el tor pla en $\mathbb R^3$, íntimament relacionada amb el teorema de Nash-Kuiper. 
\end{itemize}

\subsection*{Estructura de la mem\`oria}
El treball es divideix en cinc capítols, relacionats amb els objectius exposats a l'apartat anterior, el primer d'ells sent aquesta introducció que esteu llegint. El capítol 2 comença amb un breu recordatori de la definició de les classes de regularitat de funcions, per després centrar-se en la geometria diferencial i Riemanniana. Definirem els conceptes necessaris per a la resta del treball, enunciant i demostrant els resultats més importants. També enunciarem alguns teoremes importants, com el teorema de Sard i el teorema egregi de Gauss, sense donar-ne la demostració. Al final del capítol es discutirà la impossibilitat d'encabir isomètricament el tor pla en $\mathbb R^3$. El capítol 3 seguirà la demostració completa d'un dels dos teoremes demostrats recentment per \cite{schwartz2024} sobre la cinta de Möbius, enunciant l'altre resultat i un contraexemple $C^0$ de l'acordió de Möbius. El capítol 4 seguirà l'article original de Nash en què demostra els seus teoremes sobre encabiments i immersions $C^1$. Algunes de les demostracions d'aquest capítol s'han reestructurat i fet més explícites que en l'article original. Acabarem aquest capítol explicant el punt clau del refinament de Kuiper. Finalment, al capítol 5 exposarem el treball de \cite{borrelli2013} per trobar un encabiment isomètric $C^1$ del tor pla en $\mathbb R^3$, sense donar massa detalls en les demostracions. 
\subsection*{Glossari de termes traduits al català}
Totes les fonts utilitzades en la redacció d'aquesta memòria són en anglès, de manera que ha calgut traduir alguns termes tècnics. Si bé la majoria d'aquests tenen traducció estàndard al català, alguns d'ells no en tenen. El cas més important per aquest treball és el del terme \textit{embedding}, que sovint s'escriu en anglès i en cursiva, però aquí apareixerà com \textit{encabiment}. A continuació introduïm un petit glossari d'alguns termes anglesos i la traducció per la qual hem optat en aquest treball. 

\begin{center}
    \begin{tabular}{ll}
    \hline
    \textbf{Terme anglès} & \textbf{Traducció al català} \\
    \hline
    \textit{Bend} & \textit{Plec} \\
    \textit{Embedding} & \textit{Encabiment} \\
    \textit{Exhaustion function} & \textit{Funció d'exhauriment} \\
    \textit{Flatness} & \textit{Planor} \\
    \textit{Isometric default} & \textit{Defecte isomètric} \\
    \textit{Loops} & \textit{Voltes} \\
    \textit{Pre-bend} & \textit{Pre-plec} \\
    \textit{Smooth} & \textit{Suau} \\
    \textit{Stage} & \textit{Etapa} \\
    \textit{Step} & \textit{Pas} \\
    \textit{Strain} & \textit{Corrugació} \\
    \hline
    \end{tabular}
\end{center}
\newpage
\pagenumbering{arabic}
\setcounter{page}{1}
