\chapter{Geometria diferencial i suavitat}


{\color{blue} La ide d'aquest capítol ha de ser la següent:
\begin{itemize}
    \item Explicar temes de varietats topològiques.
    \item Parlar sobre geometria Riemanniana.
    \item Definir immersions i encabiments.
    \item Parlar de la suavitat de les immersions i encabiments.
    \item Enunciar i demostrar el tema de que qualsevol encabiment suau del tor ha de tenir algun punt el·líptic. A la pàgina 9 del llibre del Tor es comenta.

\end{itemize}
}


\section{Introducció al capítol}
En aquest capítol ens ocuparem de definir i comentar algunes de les nocions bàsiques de geometria diferencial i de varietats diferenciables i suaus. La referència principal per aquest capítol és \cite{lee2013}.\\
A un nivell intuïtiu, les varietats suaus (\textit{smooth manifolds}) són espais topològics que, localment, es poden veure com espais euclidians $\mathbb R^n$, i tals que hi sigui possible fer càlcul infinitesimal. Si bé és fàcil entendre la noció de suavitat (\textit{smoothness}) en els exemples senzills de corbes i superfícies immerses en $\mathbb R^3$, cal tenir en compte que no és cap requeriment per a una varietat suau que sigui immersa en un espai ambient $\mathbb R^n$, sinó que s'haurà de treballar en termes intrínsecs. Aquest fet és molt rellevant en una de les aplicacions més interessants de la geometria diferencial, la teoria de la relativitat general, on l'espai-temps es modela com una varietat diferenciable de dimensió 4 i on no hi ha cap espai ambient en el sentit clàssic. Un altra subtilesa és que la noció de suavitat no pot ser una propietat purament topològica, ja que no és preservada per homeomorfismes. L'exemple més evident és el d'un cercle i un quadrat, que són homeomorfs en $\mathbb R^2$, però el cercle és suau mentre que el quadrat no ho és.
{\color{blue} Amb el que escriuré ara, la referència principal passa a ser el Warner, \cite{warner1983}.}





\newpage
\section{Definicions i propietats bàsiques}
\begin{defi}
    Siguin $U\subseteq\mathbb R^n$ un conjunt obert i $f:U\to\mathbb R$ una funció real contínua.
    Diem que $f$ és \textbf{ $k$-vegades derivable contínuament}, o \textbf{de classe $C^k(U)$}, amb $k\in\mathbb N_0$, si totes les derivades parcials d'ordre $k$, \begin{equation*}
        \frac{\partial^k f}{\partial r_1^{\alpha_1}\cdots\partial r_n^{\alpha_n}}\quad\text{tal que }\sum_{i=1}^n\alpha_i = k,
    \end{equation*} existeixen i són contínues en $U$.

    Si $g:U\to\mathbb R^m$ és una aplicació contínua, diem que $g$ és \textbf{ $k$-vegades derivable contínuament}, o \textbf{de classe $C^k(U)$}, si totes les seves components $g_i:U\to\mathbb R$ són $k$-vegades derivables contínuament.
\end{defi}
\begin{nota}
    No indicarem el domini en què una aplicació és de classe $C^k$ quan el domini sigui clar pel context.
\end{nota}

\begin{obss}
\end{obss}
\begin{itemize}
    \item Una aplicació $f$ és $0$ vegades derivable contínuament si i només si $f$ és contínua. 
    \item Si $f$ és $k$ vegades derivable contínuament, aleshores també és $j$ vegades derivable contínuament per $0\le j\le k$.
\end{itemize}

\begin{defi}
    Direm que una aplicació és \textbf{suau} o \textbf{de classe $C^\infty$} si és infinitament derivable, és a dir, si és $k$-vegades derivable contínuament per a tot $k\in\mathbb N_0$.
\end{defi}

\begin{defi}
    
    Siguin $U\subseteq \mathbb R^n$ un obert i $f:U\to\mathbb R$ una funció de classe $C^k(U)$. Definim la \textbf{norma $\|\cdot\|_{C^k(U)}$} de $f$ com
    \begin{equation*}
        \|f\|_{C^k(U)} := \sum_{i = 0}^k \sup_{x\in U} \left\| f^{(i)}(x) \right\|.
    \end{equation*}
    Per una aplicació $g:U\to\mathbb{R}^m$ de classe $C^k(U)$, definim la \textbf{norma $\|\cdot\|_{C^k(U)}$} de $g$ com
    \begin{equation*}
        \|g\|_{C^k(U)} := \sum_{|\alpha| \leq k} \sup_{x\in U} \left\| \partial^\alpha f(x) \right\|,
    \end{equation*}
    on $\alpha = (\alpha_1, \dots, \alpha_n)$, $|\alpha| = \alpha_1 + \dots + \alpha_n$, i
    \[
    \partial^\alpha f(x) := \frac{\partial^{|\alpha|} f(x)}{\partial x_1^{\alpha_1} \dots \partial x_n^{\alpha_n}}.
    \]




\end{defi}

\begin{defi} 
    Sigui $M$ un espai topològic {\color{blue} Cal definició d'espai topològic?}. Diem que $M$ és una \textbf{varietat topològica de dimensió $n$} si es compleixen les propietats següents:
    \begin{itemize}
        \item $M$ és \underline{Hausdorff}, és a dir, si per a cada $p,q\in M$ amb $p\neq q$ existeixen entorns oberts $U\subseteq M$ i $V\subseteq M$ de $p$ i $q$ respectivament tals que $U\cap V = \emptyset$,
        \item $M$ verifica el \underline{segon axioma de numerabilitat}, és a dir, existeix una base numerable de la topologia de $M$,
        \item $M$ és \underline{localment homeomorf a $\mathbb R^n$}, és a dir, per a cada $p\in M$ existeix un entorn obert $U\subseteq M$ de $p$ que és homeomorf a un obert de $\mathbb R^n$.
    \end{itemize}
\end{defi}

Per tal de poder descriure localment els punts de les varietats i de poder operar amb ells, és necessari introduir el concepte de carta coordenada. 

\begin{defi}
    Sigui $M$ una varietat topològica de dimensió $n$. Diem que un parell $(U,\varphi)$ és una \textbf{carta coordenada} o un \textbf{sistema de coordenades de $M$} si $U$ és un obert de $M$ i $\varphi:U\to\hat U$ és un homeomorfisme amb un obert $\hat U\subseteq\mathbb R^n$. Anomenem $U$ el \textbf{domini de la carta} i $\varphi$ l'\textbf{aplicació coordenada}. Donat un punt $p\in U$, anomenem \textbf{coordenades de $p$} respecte de la carta $(U,\varphi)$ als components de $\varphi(p)$ en la base canònica de $\mathbb R^n$.
\end{defi}
\begin{nota}
    Sovint anomenarem carta coordenada o simplement carta a l'aplicació coordenada $\varphi$.
\end{nota}

\begin{obs}
    De la definició de carta coordenada, observem que no tota varietat topològica $M$ es pot cobrir amb una única carta coordenada. Per exemple, si $M$ és homeomorf al cercle $\mathbb S^1$ amb la topologia induïda per $\mathbb R^2$, no es pot trobar cap aplicació $\varphi:M\to\mathbb R$ que sigui un homeomorfisme amb un obert de $\mathbb R$, ja que $\mathbb S^1$ és compacte.
\end{obs}

{\color{blue} Es pot posar un exemple.}

\begin{defi}
    Sigui $M$ una varietat topològica de dimensió $n$. Anomenem \textbf{estructura diferenciable de classe $C^k$} en $M$ una col·lecció $\mathcal F:=\set{(U_\alpha,\varphi_\alpha)}$ de cartes coordenades de $M$ que compleixen les propietats següents:
    \begin{itemize}
        \item $\bigcup_{\alpha\in A} U_\alpha = M$,
        \item Si $U_\alpha\cap U_\beta\neq\emptyset$, aleshores $\varphi_\beta\circ\varphi_\alpha^{-1}$ és $C^k$.
        \item $\mathcal F$ és maximal respecte de la propietat anterior, és a dir, si $\mathcal G$ és una altra estructura diferenciable de classe $C^k$ en $M$ i $\mathcal F\subseteq\mathcal G$, aleshores $\mathcal F = \mathcal G$.
    \end{itemize}
\end{defi}

\begin{defi}
    Sigui $M$ una varietat topològica de dimensió $n$. Diem que $(M, \mathcal F)$ és una \textbf{varietat diferenciable de dimensió $n$ i classe $C^k$} si $\mathcal F$ és una estructura diferenciable de classe $C^k$ en $M$.
\end{defi}

\begin{nota}
    Sovint ens referirem a $M$ com a varietat diferenciable, sense especificar-ne l'estructura diferenciable.
\end{nota}

\begin{defi}
    {\color{blue} Aquesta definició no m'acaba d'agradar. Podem mirar si canviar-la.}

    Sigui $M$ una varietat diferenciable, $U\subseteq M$ un obert i $f:U\to\mathbb R$ una funció real. Diem que $f$ és \textbf{de classe $C^k$ en $U$} si $f\circ\varphi^{-1}$ és de classe $C^k$ per tota aplicació coordenada $\varphi$ de $M$.
    Una aplicació $\psi:M\to N$ és de classe $C^k(M,N)$ si per tota $g$ definida en oberts $V$ de $N$ la composició $g\circ\psi$ és de classe $C^k$ en $V$.
\end{defi}

\subsection{Segon axioma de numerabilitat}
A continuació veurem algunes propietats de les varietats diferenciables que es desprenen del fet que verifiquen el segon axioma de numerabilitat. El que més ens interessarà serà l'existència de particions de la unitat, en el cas les varietats de classe $C^\infty$.{\color{blue} Potser hauríem de deixar més clar en quin moment estem parlant de varietats de classe $C^\infty$ i quan no.}

\begin{defi}
    Sigui $M$ una varietat diferenciable. Anomenem \textbf{recobriment de $W\subseteq M$} a una col·lecció $\set{U_\alpha}$ de subconjunts de $M$ tals que $W = \bigcup_{\alpha\in A} U_\alpha$. Diem que el recobriment és un \textbf{recobriment per oberts} si tots els $U_\alpha$ són oberts, i un \textbf{recobriment per tancats} si tots els $U_\alpha$ són tancats.

    Donat un recobriment $\set{U_\alpha}$ de $W\subseteq M$, diem que $\set{V_\beta}$ n'és un \textbf{refinament} si per tot $\beta$ existeix un $\alpha$ tal que $V_\beta\subseteq U_\alpha$ i $\bigcup_{\beta\in B} V_\beta = \bigcup_{\alpha\in A} U_\alpha$.

    Diem que un recobriment $\set{U_\alpha}$ de $W\subseteq M$ és \textbf{localment finit} si per a cada $p\in W$ existeix un entorn $V$ de $p$ en $M$ tal que $V\cap U_\alpha = \emptyset$ per a tot $\alpha$ excepte un nombre finit. Diem que una varietat diferenciable és \textbf{paracompacta} si qualsevol recobriment per oberts té un refinament localment finit.
\end{defi}

\begin{defi}
    Sigui $M$ una varietat diferenciable suau.Una \textbf{partició de la unitat en $M$} és una col·lecció $\set{\varphi_i}_{i\in I}$ de funcions reals de classe $C^\infty(M)$ tals que:
    \begin{itemize}
        \item $0\leq\varphi_i(p)\leq 1$ per a tot $i\in I$ i $p\in M$,
        \item $\sum_{i\in I}\varphi_i(p) = 1$ per a tot $p\in M$,
        \item El conjunt de suports $\set{\text{supp}(\varphi_i)}$ és localment finit, on el \textbf{suport} d'una funció és l'adherència del conjunt de punts del seu domini on la funció no és $0$.
    \end{itemize}
    Diem que la partició de la unitat és \textbf{subordinada al recobriment} $\set{U_\alpha}$ si per a cada $i\in I$ existeix un $\alpha$ tal que $\text{supp}(\varphi_i)\subseteq U_\alpha$.
\end{defi}

\begin{lema}\label{lema:paracompact}
    Sigui $X$ un espai topològic localment compacte (és a dir, tal que tot punt de $X$ té un entorn compacte), Hausdorff i tal que verifica el segon axioma de numerabilitat. Aleshores $X$ és paracompacte, i cada recobriment per oberts de $X$ té un refinament numerable i localment finit per oberts d'adherència compacta.
\end{lema}
{\color{green!50!black} \textit{Prova.} Com $X$ verifica el segon axioma de numerabilitat, existeix una base numerable de la topologia de $X$. Com $X$ és localment compacte, podem prendre d'aquesta base numerable els conjunts amb adherència compacta, i pel fet que $X$ és Hausdorff, aquesta col·lecció de subconjunts serà una base en si mateixa. Sigui $\set{U_i}_{i\in I}$ aquesta base.

Sigui $G_1 := U_1$, i suposem que hem definit un cert $G_k=U_1\cup\cdots\cup U_{j_k}$. Sigui $j_{k+1}$ l'enter més petit tal que sigui estrictament més gran que $j_k$ i tal que 
\begin{equation*}
    \overline{G_k}\subseteq \bigcup_{i = 1}^{j_{k+1}} U_i,
\end{equation*}
i definim 
\begin{equation*}
    G_{k+1} := G_k\cup U_{j_{k+1}}.
\end{equation*}
D'aquesta manera, obtenim inductivament una successió de conjunts oberts $G_k$ tals que per tot $k$ tenim que
\begin{enumerate}
    \item $\overline{G_k}$ és compacte,
    \item $\overline{G_k}\subseteq G_{k+1}$,
    \item $X = \bigcup_{k\in\mathbb N} G_k$.
\end{enumerate}
{\color{blue} Ojito aqui que hem canviat la manera en què enumerem els $G_k$ a $G_i$. El millor seria que els escrivissim de manera consistent.}
Ara sigui $\set{U_\alpha:\alpha\in A}$ un recobriment per oberts qualsevol. El conjunt $\overline{G_{i+1}}\setminus G_{i-1}$ és compacte i contingut en l'obert $G_{i+1}\setminus \overline{G_{i-2}}$. Per tot $i\ge3$, podem escollir un subrecobriment finit del recobriment per oberts $U_\alpha\cap(G_{i+1}\setminus \overline{G_{i-2}})$ de $\overline{G_i}\setminus G_{i-1}$, i es pot escollir un subrecobriment finit del recobriment per oberts $U_\alpha\cap G_3$ del conjunt compacte $\overline{G_2}$. Aquesta nova col·lecció serà un refinament numerable i localment finit per oberts d'adherència compacta del recobriment $\set{U_\alpha}$, com volíem veure. \qed
}
\begin{lema}
    Existeix una funció real no-negativa $\varphi:\mathbb R^n\to\mathbb R$ de classe $C^\infty$ en que és igual a 1 en $[-1,1]^n$ i $0$ en el complementari de $(-2,2)^n$.
\end{lema}
{\color{green!50!black} \textit{Prova.} Definim la funció com un producte de la forma
\begin{equation}\label{eq:phi}
    \varphi = (h\circ x_1)\cdots(h\circ x_n),
\end{equation}
on
\begin{equation*}
    h(t) = g(t+2)g(2-t),
\end{equation*}
\begin{equation*}
    g(t) = \frac{f(t)}{f(t) + f(1-t)}
\end{equation*}
i 
\begin{equation*}
    f(t) = \begin{cases}
        e^{-1/t} & \text{si } t > 0, \\
        0 & \text{si } t\le 0.
    \end{cases}
\end{equation*}
\qed
}
\begin{teo}[\textbf{Existència de particions de la unitat}]
    {\color{blue}Fer referència a això a la demostració del Teorema de Nash.}

    Sigui $M$ una varietat diferenciable de classe $C^\infty$ i $\set{U_\alpha:\alpha\in A}$ un recobriment per oberts de $M$. Aleshores, existeix una partició de la unitat numberable $\set{\varphi_i:i\in I}$ subordinada al recobriment $\set{U_\alpha}$ amb suports $\set{\text{supp}\varphi_i}$ compactes. A més, si no exigim que els suports siguin compactes, existeix una partició de la unitat $\set{\varphi_\alpha}$ subordinada al recobriment $\set{U_\alpha}$ amb $\text{supp}\varphi_\alpha \subseteq U_\alpha$ per a tot $\alpha\in A$, amb com a molt un conjunt numerable dels $\varphi_\alpha$ no idènticament zero.
\end{teo}
{\color{green!50!black} \textit{Prova.} 
Sigui $\set{G_i}$ un recobriment com el definit a la demostració del lema \ref{lema:paracompact} i definim $G_0=\emptyset$. Per cada punt $p\in M$, sigui $i_p$ l'enter més gran tal que $p\in M\setminus \overline{G_{i_p}}$. Escollim un $\alpha_p$ tal que $p\in U_{\alpha_p}$ i sigui $(V,\tau)$ una carta coordenada centrada en $p$ (és a dir, tal que $\tau(p)=0$ {\color{blue} Comprovar que aquesta és la definició correcta molt ràpid}) i tal que $V\subseteq U_{\alpha_p}\cap(G_{i_p+2}\setminus \overline{G_{i_p}})$ i $[-2,2]^n\subseteq\tau(V)$.

Definim 
\begin{equation*}
    \psi_p = \begin{cases}
        \varphi\circ\tau & \text{a } V, \\
        0 & \text{fora de } V.
    \end{cases}
\end{equation*}
on $\varphi$ és tal com l'hem definit a l'equació \ref{eq:phi}. Aleshores, $\psi_p$ és una funció de classe $C^\infty$ en $M$ que val $1$ en un entorn $W_p$ de $p$ i té suport compacte en $V\subseteq U_{\alpha_p}\cap(G_{i_p+2}\setminus \overline{G_{i_p}})$. Per cada $i\ge1$, escollim un nombre finit de punts $p\in M$ tals que els respectius $W_p$ siguin un recobriment de $\overline{G_i}\setminus G_{i-1}$. Podem escollir un ordre qualsevol per les funcions corresponents $\psi_p$ per obtenir una successió $\set{\psi_j}$, i els seus suports formen una col·lecció localment finita de subconjunts de $M$. Amb això veiem que la funció 
\begin{equation*}
    \psi = \sum_{j=1}^\infty \psi_j
\end{equation*}
és positiva i de classe $C^\infty$ en $M$. Ara, per tot $i=1,2,\dots$ definim 
\begin{equation*}
    \varphi_i = \frac{\psi_i}{\psi}.
\end{equation*}
Les funcions $\varphi_i$ formen una partició de la unitat subordinada al recobriment $\set{U_\alpha}$ amb suports compactes.

Si definim $\varphi_\alpha$ tals que siguin idènticament zero si cap $\varphi_i$ té suport en $U_\alpha$, i en cas contrari que siguin la suma de totes les $\varphi_i$ que hi tenen suport, aleshores tenim que $\varphi_\alpha$ formen una partició de la unitat subordinada al recobriment $\set{U_\alpha}$ amb com a molt un conjunt numerable dels $\varphi_\alpha$ no idènticament zero.

Veiem que el suport de cada $\varphi_\alpha$ és contingut en $U_\alpha$, ja que si $\mathcal A$ és una col·lecció localment finita de conjunts tancats, aleshores $\overline{\bigcup_{A\in \mathcal A} A} = \bigcup_{A\in \mathcal A} A$. Ara bé, observem que en aquest cas el suport de $\varphi_\alpha$ no és necessàriament compacte.
\qed
}
\begin{corol}
    Sigui $G$ un obert d'una varietat diferenciable de classe $C^\infty$ i $A\subseteq G$ un subconjunt tancat. Aleshores, existeix una funció $\varphi:M\to\mathbb R$ de classe $C^\infty$ en $M$ tal que
    \begin{enumerate}
        \item $0\leq\varphi(p)\leq 1$ per a tot $p\in M$,
        \item $\varphi(p) = 1$ per a tot $p\in A$,
        \item $\text{supp}\varphi\subseteq G$.
    \end{enumerate}
\end{corol}
{
\color{green!50!black} \textit{Prova.} 
Existeix una partició de la unitat $\set{\varphi, \psi}$ subordinada al recobriment $\set{G, M\setminus A}$ de $M$ amb $\text{supp}\varphi\subseteq G$ i $\text{supp}\psi\subseteq M\setminus A$. $\varphi$ és, per tant, la funció desitjada. \qed
}

{\color{red} ÉS MOLT IMPORTANT QUE DESPRÉS MIREM LA PÀGINA 24 DEL LEE, ON DEFINEIX TAMBÉ VARIETATS AMB FRONTERA}

\subsection{Vectors tangents}
Un vector en $\mathbb R^n$ es pot pensar com un operador lineal sobre funcions reals diferenciables. En concret, donada una funció $f$ diferenciable en un punt $p\in\mathbb R^n$, el vector $v$ assigna a $f$ un valor real que és la derivada direccional de $f$ en la direcció i sentit de $v$ a $p$,
\begin{equation*}
    v(f) = v_1\frac{\partial f}{\partial x_1}\Big|_p + \cdots + v_n\frac{\partial f}{\partial x_n}\Big|_p,
\end{equation*}
amb les propietats de linealitat esperades,
\begin{equation*}
    v(f+g) = v(f) + v(g),
\end{equation*}
\begin{equation*}
    v(\lambda f) = \lambda v(f),
\end{equation*}
i la propietat de Leibniz,
\begin{equation*}
    v(fg) = v(f)g(p) + f(p)v(g).
\end{equation*}

És evident que volem un anàleg a aquesta definició que sigui útil en el context de varietats diferenciables, per tal d'aprofitar el fet que aquestes són espais localment similars a $\mathbb R^n$. 
\begin{defi}
    Sigui $M$ una varietat diferenciable i $(U,\varphi)$ una carta coordenada de $M$. Un \textbf{vector tangent} a $M$ en un punt $p\in M$ és una aplicació que assigna a cada funció $f$ de classe $C^k$ en un entorn de $p$ un valor real $v(f)$ tal que existeixi una col·lecció de nombres reals $(a_1,\dots,a_n)$ tals que
    \begin{equation*}
        v(f) = \sum_{i=1}^n a_i\frac{\partial (f\circ\varphi^{-1})}{\partial x_i}\Big|_{\varphi(p)}.
    \end{equation*}
    Anomenem el conjunt de tots aquests vectors tangents l'\textbf{espai tangent} a $M$ en $p$ i el denotem per $T_pM$.
\end{defi}
{\color{blue} Hi ha al Lee una mica més de detall i demostra que té la dimensió que toca i tal, es pot mirar.}

Una eina essencial per treballar amb vectors tangents serà el concepte de \textbf{diferencial} d'una aplicació diferenciable entre varietats.
{\color{blue} Crec que el concepte d'aplicació diferenciable entre varietats no l'hem definit encara. A més, serà important definir-lo pel cas $C^k$ ja que el necessitarem.}
\begin{defi}
    {\color{blue} La necessitem en el cas $C^k$ ja que és important per les definicions de la subsecció següent}Sigui $M$ i $N$ varietats diferenciables suaus, $F:M\to N$ una aplicació diferenciable i $p\in M$. Anomenem \textbf{diferencial de $F$ en $p$} l'aplicació
    \begin{equation*}
        dF_p:T_pM\to T_{F(p)}N
    \end{equation*}
    que assigna a cada vector tangent $v\in T_pM$ el vector tangent $dF_p(v)\in T_{F(p)}N$ que compleix
    \begin{equation*}
        dF_p(v)(f) = v(f\circ F),
    \end{equation*}
    per a tota $f\in C^\infty(N)$.
\end{defi}
\begin{defi}
    Diem que una aplicació $F$ és \textbf{no-singular} en $p\in M$ si $dF_p$ no és singular, és a dir, si el seu nucli és $\set0$.
\end{defi}
\subsection{Subvarietats, immersions, encabiments i difeomorfismes}
\begin{defi}
    Sigui $\psi:M\to N$ una aplicació diferenciable {\color{blue} $C^\infty$}.
    \begin{enumerate}
        \item Diem que $\psi$ és una \textbf{immersió} si $d\psi_p$ és no-singular per a tot $p\in M$.
        \item Diem que $\psi$ és un \textbf{encabiment} si és una immersió injectiva i un homeomorfisme sobre la seva imatge.
        \item Diem que $\psi$ és un \textbf{difeomorfisme} si és una injectiva amb inversa diferenciable {\color{blue} $C^\infty$}.
        \item El parell $(M, \psi)$ és una \textbf{subvarietat de $N$} si $\psi$ és una immersió injectiva.
    \end{enumerate}
\end{defi}



\newpage