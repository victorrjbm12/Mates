\chapter{Geometria diferencial i suavitat}


{\color{blue} La ide d'aquest capítol ha de ser la següent:
\begin{itemize}
    \item Explicar temes de varietats topològiques.
    \item Parlar sobre geometria Riemanniana.
    \item Definir immersions i encabiments.
    \item Parlar de la suavitat de les immersions i encabiments.
    \item Enunciar i demostrar el tema de que qualsevol encabiment suau del tor ha de tenir algun punt el·líptic. A la pàgina 9 del llibre del Tor es comenta.

\end{itemize}
}
{\color{red} El que haurem de fer per a que tot tingui sentit un cop estigui acabat és assegurar-nos que anomenem $x$ a les coordenades normals i $z$ a les coordenades de la carta.}

\section{Introducció al capítol}
En aquest capítol ens ocuparem de definir i comentar algunes de les nocions bàsiques de geometria diferencial i de varietats diferenciables i suaus. La referència principal per aquest capítol és \cite{lee2013}.\\
A un nivell intuïtiu, les varietats suaus (\textit{smooth manifolds}) són espais topològics que, localment, es poden veure com espais euclidians $\mathbb R^n$, i tals que hi sigui possible fer càlcul infinitesimal. Si bé és fàcil entendre la noció de suavitat (\textit{smoothness}) en els exemples senzills de corbes i superfícies immerses en $\mathbb R^3$, cal tenir en compte que no és cap requeriment per a una varietat suau que sigui immersa en un espai ambient $\mathbb R^n$, sinó que s'haurà de treballar en termes intrínsecs. Aquest fet és molt rellevant en una de les aplicacions més interessants de la geometria diferencial, la teoria de la relativitat general, on l'espai-temps es modela com una varietat diferenciable de dimensió 4 i on no hi ha cap espai ambient en el sentit clàssic. Un altra subtilesa és que la noció de suavitat no pot ser una propietat purament topològica, ja que no és preservada per homeomorfismes. L'exemple més evident és el d'un cercle i un quadrat, que són homeomorfs en $\mathbb R^2$, però el cercle és suau mentre que el quadrat no ho és.
{\color{blue} Amb el que escriuré ara, la referència principal passa a ser el Warner, \cite{warner1983}.}





\newpage
\section{Definicions i propietats bàsiques}
\begin{defi}
    Siguin $U\subseteq\mathbb R^n$ un conjunt obert i $f:U\to\mathbb R$ una funció real contínua.
    Diem que $f$ és \textbf{ $k$-vegades derivable contínuament}, o \textbf{de classe $C^k(U)$}, amb $k\in\mathbb N_0$, si totes les derivades parcials d'ordre $k$, \begin{equation*}
        \frac{\partial^k f}{\partial r_1^{\alpha_1}\cdots\partial r_n^{\alpha_n}}\quad\text{tal que }\sum_{i=1}^n\alpha_i = k,
    \end{equation*} existeixen i són contínues en $U$.

    Si $g:U\to\mathbb R^m$ és una aplicació contínua, diem que $g$ és \textbf{ $k$-vegades derivable contínuament}, o \textbf{de classe $C^k(U)$}, si totes les seves components $g_i:U\to\mathbb R$ són $k$-vegades derivables contínuament.
\end{defi}
\begin{nota}
    No indicarem el domini en què una aplicació és de classe $C^k$ quan el domini sigui clar pel context.
\end{nota}

\begin{obss}
\end{obss}
\begin{itemize}
    \item Una aplicació $f$ és $0$ vegades derivable contínuament si i només si $f$ és contínua. 
    \item Si $f$ és $k$ vegades derivable contínuament, aleshores també és $j$ vegades derivable contínuament per $0\le j\le k$.
\end{itemize}

\begin{defi}
    Direm que una aplicació és \textbf{suau} o \textbf{de classe $C^\infty$} si és infinitament derivable, és a dir, si és $k$-vegades derivable contínuament per a tot $k\in\mathbb N_0$.
\end{defi}

\begin{defi}
    
    Siguin $U\subseteq \mathbb R^n$ un obert i $f:U\to\mathbb R$ una funció de classe $C^k(U)$. Definim la \textbf{norma $\|\cdot\|_{C^k(U)}$} de $f$ com
    \begin{equation*}
        \|f\|_{C^k(U)} := \sum_{i = 0}^k \sup_{x\in U} \left\| f^{(i)}(x) \right\|.
    \end{equation*}
    Per una aplicació $g:U\to\mathbb{R}^m$ de classe $C^k(U)$, definim la \textbf{norma $\|\cdot\|_{C^k(U)}$} de $g$ com
    \begin{equation*}
        \|g\|_{C^k(U)} := \sum_{|\alpha| \leq k} \sup_{x\in U} \left\| \partial^\alpha f(x) \right\|,
    \end{equation*}
    on $\alpha = (\alpha_1, \dots, \alpha_n)$, $|\alpha| = \alpha_1 + \dots + \alpha_n$, i
    \[
    \partial^\alpha f(x) := \frac{\partial^{|\alpha|} f(x)}{\partial x_1^{\alpha_1} \dots \partial x_n^{\alpha_n}}.
    \]




\end{defi}

\begin{defi} 
    Sigui $M$ un espai topològic {\color{blue} Cal definició d'espai topològic?}. Diem que $M$ és una \textbf{varietat topològica de dimensió $n$} si es compleixen les propietats següents:
    \begin{itemize}
        \item $M$ és \underline{Hausdorff}, és a dir, si per a cada $p,q\in M$ amb $p\neq q$ existeixen entorns oberts $U\subseteq M$ i $V\subseteq M$ de $p$ i $q$ respectivament tals que $U\cap V = \emptyset$,
        \item $M$ verifica el \underline{segon axioma de numerabilitat}, és a dir, existeix una base numerable de la topologia de $M$,
        \item $M$ és \underline{localment homeomorf a $\mathbb R^n$}, és a dir, per a cada $p\in M$ existeix un entorn obert $U\subseteq M$ de $p$ que és homeomorf a un obert de $\mathbb R^n$.
    \end{itemize}
\end{defi}

Per tal de poder descriure localment els punts de les varietats i de poder operar amb ells, és necessari introduir el concepte de carta coordenada. 

\begin{defi}
    Sigui $M$ una varietat topològica de dimensió $n$. Diem que un parell $(U,\varphi)$ és una \textbf{carta coordenada} o un \textbf{sistema de coordenades de $M$} si $U$ és un obert de $M$ i $\varphi:U\to\hat U$ és un homeomorfisme amb un obert $\hat U\subseteq\mathbb R^n$. Anomenem $U$ el \textbf{domini de la carta} i $\varphi$ l'\textbf{aplicació coordenada}. Donat un punt $p\in U$, anomenem \textbf{coordenades de $p$} respecte de la carta $(U,\varphi)$ als components de $\varphi(p)$ en la base canònica de $\mathbb R^n$.
\end{defi}
\begin{nota}
    Sovint anomenarem carta coordenada o simplement carta a l'aplicació coordenada $\varphi$.
\end{nota}

\begin{obs}
    De la definició de carta coordenada, observem que no tota varietat topològica $M$ es pot cobrir amb una única carta coordenada. Per exemple, si $M$ és homeomorf al cercle $\mathbb S^1$ amb la topologia induïda per $\mathbb R^2$, no es pot trobar cap aplicació $\varphi:M\to\mathbb R$ que sigui un homeomorfisme amb un obert de $\mathbb R$, ja que $\mathbb S^1$ és compacte.
\end{obs}

{\color{blue} Es pot posar un exemple.}

\begin{defi}
    Sigui $M$ una varietat topològica de dimensió $n$. Anomenem \textbf{estructura diferenciable de classe $C^k$} en $M$ una col·lecció $\mathcal F:=\set{(U_\alpha,\varphi_\alpha)}$ de cartes coordenades de $M$ que compleixen les propietats següents:
    \begin{itemize}
        \item $\bigcup_{\alpha\in A} U_\alpha = M$,
        \item Si $U_\alpha\cap U_\beta\neq\emptyset$, aleshores $\varphi_\beta\circ\varphi_\alpha^{-1}$ és $C^k$.
        \item $\mathcal F$ és maximal respecte de la propietat anterior, és a dir, si $\mathcal G$ és una altra estructura diferenciable de classe $C^k$ en $M$ i $\mathcal F\subseteq\mathcal G$, aleshores $\mathcal F = \mathcal G$.
    \end{itemize}
\end{defi}

\begin{defi}
    Sigui $M$ una varietat topològica de dimensió $n$. Diem que $(M, \mathcal F)$ és una \textbf{varietat diferenciable de dimensió $n$ i classe $C^k$} si $\mathcal F$ és una estructura diferenciable de classe $C^k$ en $M$.
\end{defi}

\begin{nota}
    Sovint ens referirem a $M$ com a varietat diferenciable, sense especificar-ne l'estructura diferenciable.
\end{nota}

\begin{defi}
    {\color{blue} Aquesta definició no m'acaba d'agradar. Podem mirar si canviar-la.}

    Sigui $M$ una varietat diferenciable, $U\subseteq M$ un obert i $f:U\to\mathbb R$ una funció real. Diem que $f$ és \textbf{de classe $C^k$ en $U$} si $f\circ\varphi^{-1}$ és de classe $C^k$ per tota aplicació coordenada $\varphi$ de $M$.
    Una aplicació $\psi:M\to N$ és de classe $C^k(M,N)$ si per tota $g$ definida en oberts $V$ de $N$ la composició $g\circ\psi$ és de classe $C^k$ en $V$.
\end{defi}

\subsection{Segon axioma de numerabilitat}
A continuació veurem algunes propietats de les varietats diferenciables que es desprenen del fet que verifiquen el segon axioma de numerabilitat. El que més ens interessarà serà l'existència de particions de la unitat, en el cas les varietats de classe $C^\infty$.{\color{blue} Potser hauríem de deixar més clar en quin moment estem parlant de varietats de classe $C^\infty$ i quan no.}

\begin{defi}
    Sigui $M$ una varietat diferenciable. Anomenem \textbf{recobriment de $W\subseteq M$} a una col·lecció $\set{U_\alpha}$ de subconjunts de $M$ tals que $W = \bigcup_{\alpha\in A} U_\alpha$. Diem que el recobriment és un \textbf{recobriment per oberts} si tots els $U_\alpha$ són oberts, i un \textbf{recobriment per tancats} si tots els $U_\alpha$ són tancats.

    Donat un recobriment $\set{U_\alpha}$ de $W\subseteq M$, diem que $\set{V_\beta}$ n'és un \textbf{refinament} si per tot $\beta$ existeix un $\alpha$ tal que $V_\beta\subseteq U_\alpha$ i $\bigcup_{\beta\in B} V_\beta = \bigcup_{\alpha\in A} U_\alpha$.

    Diem que un recobriment $\set{U_\alpha}$ de $W\subseteq M$ és \textbf{localment finit} si per a cada $p\in W$ existeix un entorn $V$ de $p$ en $M$ tal que $V\cap U_\alpha = \emptyset$ per a tot $\alpha$ excepte un nombre finit. Diem que una varietat diferenciable és \textbf{paracompacta} si qualsevol recobriment per oberts té un refinament localment finit.
\end{defi}

\begin{defi}
    Sigui $M$ una varietat diferenciable suau.Una \textbf{partició de la unitat en $M$} és una col·lecció $\set{\varphi_i}_{i\in I}$ de funcions reals de classe $C^\infty(M)$ tals que:
    \begin{itemize}
        \item $0\leq\varphi_i(p)\leq 1$ per a tot $i\in I$ i $p\in M$,
        \item $\sum_{i\in I}\varphi_i(p) = 1$ per a tot $p\in M$,
        \item El conjunt de suports $\set{\text{supp}(\varphi_i)}$ és localment finit, on el \textbf{suport} d'una funció és l'adherència del conjunt de punts del seu domini on la funció no és $0$.
    \end{itemize}
    Diem que la partició de la unitat és \textbf{subordinada al recobriment} $\set{U_\alpha}$ si per a cada $i\in I$ existeix un $\alpha$ tal que $\text{supp}(\varphi_i)\subseteq U_\alpha$.
\end{defi}

\begin{lema}\label{lema:paracompact}
    Sigui $X$ un espai topològic localment compacte (és a dir, tal que tot punt de $X$ té un entorn compacte), Hausdorff i tal que verifica el segon axioma de numerabilitat. Aleshores $X$ és paracompacte, i cada recobriment per oberts de $X$ té un refinament numerable i localment finit per oberts d'adherència compacta.
\end{lema}
{\color{green!50!black} \textit{Prova.} Com $X$ verifica el segon axioma de numerabilitat, existeix una base numerable de la topologia de $X$. Com $X$ és localment compacte, podem prendre d'aquesta base numerable els conjunts amb adherència compacta, i pel fet que $X$ és Hausdorff, aquesta col·lecció de subconjunts serà una base en si mateixa. Sigui $\set{U_i}_{i\in I}$ aquesta base.

Sigui $G_1 := U_1$, i suposem que hem definit un cert $G_k=U_1\cup\cdots\cup U_{j_k}$. Sigui $j_{k+1}$ l'enter més petit tal que sigui estrictament més gran que $j_k$ i tal que 
\begin{equation*}
    \overline{G_k}\subseteq \bigcup_{i = 1}^{j_{k+1}} U_i,
\end{equation*}
i definim 
\begin{equation*}
    G_{k+1} := G_k\cup U_{j_{k+1}}.
\end{equation*}
D'aquesta manera, obtenim inductivament una successió de conjunts oberts $G_k$ tals que per tot $k$ tenim que
\begin{enumerate}
    \item $\overline{G_k}$ és compacte,
    \item $\overline{G_k}\subseteq G_{k+1}$,
    \item $X = \bigcup_{k\in\mathbb N} G_k$.
\end{enumerate}
{\color{blue} Ojito aqui que hem canviat la manera en què enumerem els $G_k$ a $G_i$. El millor seria que els escrivissim de manera consistent.}
Ara sigui $\set{U_\alpha:\alpha\in A}$ un recobriment per oberts qualsevol. El conjunt $\overline{G_{i+1}}\setminus G_{i-1}$ és compacte i contingut en l'obert $G_{i+1}\setminus \overline{G_{i-2}}$. Per tot $i\ge3$, podem escollir un subrecobriment finit del recobriment per oberts $U_\alpha\cap(G_{i+1}\setminus \overline{G_{i-2}})$ de $\overline{G_i}\setminus G_{i-1}$, i es pot escollir un subrecobriment finit del recobriment per oberts $U_\alpha\cap G_3$ del conjunt compacte $\overline{G_2}$. Aquesta nova col·lecció serà un refinament numerable i localment finit per oberts d'adherència compacta del recobriment $\set{U_\alpha}$, com volíem veure. \qed
}
\begin{lema}
    Existeix una funció real no-negativa $\varphi:\mathbb R^n\to\mathbb R$ de classe $C^\infty$ en que és igual a 1 en $[-1,1]^n$ i $0$ en el complementari de $(-2,2)^n$.
\end{lema}
{\color{green!50!black} \textit{Prova.} Definim la funció com un producte de la forma
\begin{equation}\label{eq:phi}
    \varphi = (h\circ x_1)\cdots(h\circ x_n),
\end{equation}
on
\begin{equation*}
    h(t) = g(t+2)g(2-t),
\end{equation*}
\begin{equation*}
    g(t) = \frac{f(t)}{f(t) + f(1-t)}
\end{equation*}
i 
\begin{equation*}
    f(t) = \begin{cases}
        e^{-1/t} & \text{si } t > 0, \\
        0 & \text{si } t\le 0.
    \end{cases}
\end{equation*}
{\color{blue} L'Ignasi vol que expliqui bé aquesta demostració}
\qed
}
\begin{teo}[\textbf{Existència de particions de la unitat}]
    {\color{blue}Fer referència a això a la demostració del Teorema de Nash.}

    Sigui $M$ una varietat diferenciable de classe $C^\infty$ i $\set{U_\alpha:\alpha\in A}$ un recobriment per oberts de $M$. Aleshores, existeix una partició de la unitat numberable $\set{\varphi_i:i\in I}$ subordinada al recobriment $\set{U_\alpha}$ amb suports $\set{\text{supp}\varphi_i}$ compactes. A més, si no exigim que els suports siguin compactes, existeix una partició de la unitat $\set{\varphi_\alpha}$ subordinada al recobriment $\set{U_\alpha}$ amb $\text{supp}\varphi_\alpha \subseteq U_\alpha$ per a tot $\alpha\in A$, amb com a molt un conjunt numerable dels $\varphi_\alpha$ no idènticament zero.
\end{teo}
{\color{green!50!black} \textit{Prova.} 
Sigui $\set{G_i}$ un recobriment com el definit a la demostració del lema \ref{lema:paracompact} i definim $G_0=\emptyset$. Per cada punt $p\in M$, sigui $i_p$ l'enter més gran tal que $p\in M\setminus \overline{G_{i_p}}$. Escollim un $\alpha_p$ tal que $p\in U_{\alpha_p}$ i sigui $(V,\tau)$ una carta coordenada centrada en $p$ (és a dir, tal que $\tau(p)=0$ {\color{blue} Comprovar que aquesta és la definició correcta molt ràpid}) i tal que $V\subseteq U_{\alpha_p}\cap(G_{i_p+2}\setminus \overline{G_{i_p}})$ i $[-2,2]^n\subseteq\tau(V)$.

Definim 
\begin{equation*}
    \psi_p = \begin{cases}
        \varphi\circ\tau & \text{a } V, \\
        0 & \text{fora de } V.
    \end{cases}
\end{equation*}
on $\varphi$ és tal com l'hem definit a l'equació \ref{eq:phi}. Aleshores, $\psi_p$ és una funció de classe $C^\infty$ en $M$ que val $1$ en un entorn $W_p$ de $p$ i té suport compacte en $V\subseteq U_{\alpha_p}\cap(G_{i_p+2}\setminus \overline{G_{i_p}})$. Per cada $i\ge1$, escollim un nombre finit de punts $p\in M$ tals que els respectius $W_p$ siguin un recobriment de $\overline{G_i}\setminus G_{i-1}$. Podem escollir un ordre qualsevol per les funcions corresponents $\psi_p$ per obtenir una successió $\set{\psi_j}$, i els seus suports formen una col·lecció localment finita de subconjunts de $M$. Amb això veiem que la funció 
\begin{equation*}
    \psi = \sum_{j=1}^\infty \psi_j
\end{equation*}
és positiva i de classe $C^\infty$ en $M$. Ara, per tot $i=1,2,\dots$ definim 
\begin{equation*}
    \varphi_i = \frac{\psi_i}{\psi}.
\end{equation*}
Les funcions $\varphi_i$ formen una partició de la unitat subordinada al recobriment $\set{U_\alpha}$ amb suports compactes.

Si definim $\varphi_\alpha$ tals que siguin idènticament zero si cap $\varphi_i$ té suport en $U_\alpha$, i en cas contrari que siguin la suma de totes les $\varphi_i$ que hi tenen suport, aleshores tenim que $\varphi_\alpha$ formen una partició de la unitat subordinada al recobriment $\set{U_\alpha}$ amb com a molt un conjunt numerable dels $\varphi_\alpha$ no idènticament zero.

Veiem que el suport de cada $\varphi_\alpha$ és contingut en $U_\alpha$, ja que si $\mathcal A$ és una col·lecció localment finita de conjunts tancats, aleshores $\overline{\bigcup_{A\in \mathcal A} A} = \bigcup_{A\in \mathcal A} A$. Ara bé, observem que en aquest cas el suport de $\varphi_\alpha$ no és necessàriament compacte.
\qed
}
\begin{corol}
    Sigui $G$ un obert d'una varietat diferenciable de classe $C^\infty$ i $A\subseteq G$ un subconjunt tancat. Aleshores, existeix una funció $\varphi:M\to\mathbb R$ de classe $C^\infty$ en $M$ tal que
    \begin{enumerate}
        \item $0\leq\varphi(p)\leq 1$ per a tot $p\in M$,
        \item $\varphi(p) = 1$ per a tot $p\in A$,
        \item $\text{supp}\varphi\subseteq G$.
    \end{enumerate}
\end{corol}
{
\color{green!50!black} \textit{Prova.} 
Existeix una partició de la unitat $\set{\varphi, \psi}$ subordinada al recobriment $\set{G, M\setminus A}$ de $M$ amb $\text{supp}\varphi\subseteq G$ i $\text{supp}\psi\subseteq M\setminus A$. $\varphi$ és, per tant, la funció desitjada. \qed
}

{\color{red} ÉS MOLT IMPORTANT QUE DESPRÉS MIREM LA PÀGINA 24 DEL LEE, ON DEFINEIX TAMBÉ VARIETATS AMB FRONTERA}

\subsection{Vectors tangents}
Un vector en $\mathbb R^n$ es pot pensar com un operador lineal sobre funcions reals diferenciables. En concret, donada una funció $f$ diferenciable en un punt $p\in\mathbb R^n$, el vector $v$ assigna a $f$ un valor real que és la derivada direccional de $f$ en la direcció i sentit de $v$ a $p$,
\begin{equation*}
    v(f) = v_1\frac{\partial f}{\partial x_1}\Big|_p + \cdots + v_n\frac{\partial f}{\partial x_n}\Big|_p,
\end{equation*}
amb les propietats de linealitat esperades,
\begin{equation*}
    v(f+g) = v(f) + v(g),
\end{equation*}
\begin{equation*}
    v(\lambda f) = \lambda v(f),
\end{equation*}
i la propietat de Leibniz,
\begin{equation*}
    v(fg) = v(f)g(p) + f(p)v(g).
\end{equation*}

És evident que volem un anàleg a aquesta definició que sigui útil en el context de varietats diferenciables, per tal d'aprofitar el fet que aquestes són espais localment similars a $\mathbb R^n$. 
\begin{defi}
    Sigui $M$ una varietat diferenciable i $(U,\varphi)$ una carta coordenada de $M$. Un \textbf{vector tangent} a $M$ en un punt $p\in M$ és una aplicació que assigna a cada funció $f$ de classe $C^k$ en un entorn de $p$ un valor real $v(f)$ tal que existeixi una col·lecció de nombres reals $(a_1,\dots,a_n)$ tals que
    \begin{equation*}
        v(f) = \sum_{i=1}^n a_i\frac{\partial (f\circ\varphi^{-1})}{\partial x_i}\Big|_{\varphi(p)}.
    \end{equation*}
    Anomenem el conjunt de tots aquests vectors tangents l'\textbf{espai tangent} a $M$ en $p$ i el denotem per $T_pM$.
\end{defi}
{\color{blue} Hi ha al Lee una mica més de detall i demostra que té la dimensió que toca i tal, es pot mirar.}
\begin{defi}
    Sigui $M$ una varietat diferenciable. Per cada punt $p\in M$, definim \textbf{l'espai cotangent} a $p$, $T^*_pM$, com l'espai vectorial dual de $T_pM$,
    \begin{equation*}
        T^*_pM = (T_pM)^*.
    \end{equation*}
\end{defi}






Una eina essencial per treballar amb vectors tangents serà el concepte de \textbf{diferencial} d'una aplicació diferenciable entre varietats.
{\color{blue} Crec que el concepte d'aplicació diferenciable entre varietats no l'hem definit encara. A més, serà important definir-lo pel cas $C^k$ ja que el necessitarem.}
\begin{defi}
    {\color{blue} La necessitem en el cas $C^k$ ja que és important per les definicions de la subsecció següent}
    
    Siguin $M$ i $N$ varietats diferenciables suaus, $F:M\to N$ una aplicació diferenciable i $p\in M$. Anomenem \textbf{diferencial de $F$ en $p$} l'aplicació
    \begin{equation*}
        dF_p:T_pM\to T_{F(p)}N
    \end{equation*}
    que assigna a cada vector tangent $v\in T_pM$ el vector tangent $dF_p(v)\in T_{F(p)}N$ que compleix
    \begin{equation*}
        dF_p(v)(f) = v(f\circ F),
    \end{equation*}
    per a tota $f\in C^\infty(N)$.
\end{defi}
\begin{defi}
    Diem que una aplicació $F$ és \textbf{no-singular} en $p\in M$ si $dF_p$ no és singular, és a dir, si el seu nucli és $\set0$.
\end{defi}
Dualitzant l'aplicació que defineix el diferencial, podem definir el \textit{pullback} d'una funció diferenciable, que serà particularment important per a les mètriques Riemannianes.
\begin{defi}
    Siguin $M$ i $N$ varietats diferenciables suaus, $F:M\to N$ una aplicació diferenciable i $p\in M$. Anomenem \textbf{\textit{pullback} puntual de $F$ en $p$} l'aplicació
    \begin{equation*}
        dF_p^*:T_{F(p)}^*N\to T^*_pM 
    \end{equation*}
    obtinguda dualitzant el diferencial $dF_p$.
\end{defi}
Observem que el \textit{pullback} puntual està caracteritzat per la propietat
\begin{equation*}
    dF_p^*(w)(v) = w(dF_p(v)),
\end{equation*}
per a tot $v\in T_pM$ i $w\in T^*_{F(p)}N$.



Una altra eina que necessitarem per descriure la geometria local de les varietats diferenciables és el de \textbf{fibrat tangent}.
\begin{defi}
    Sigui $M$ una varietat diferenciable. El \textbf{fibrat tangent} de $M$ és la unió disjunta dels espais tangents de tots els punts de $M$, 
    \begin{equation*}
        TM = \bigcup_{p\in M} T_pM,
    \end{equation*}
    juntament amb la \textbf{projecció} $\pi:TM\to M$ que a cada vector tangent li assigna el punt de la varietat al qual és tangent,
    \begin{equation*}
        \pi(v) = p,
    \end{equation*}
    on $v\in T_pM$.
\end{defi}
De la mateixa manera, podem definir el \textbf{fibrat cotangent} de $M$ com la unió disjunta dels espais cotangents de tots els punts de $M$,
\begin{equation*}
    T^*M = \bigcup_{p\in M} T^*_pM,
\end{equation*}
junt amb la \textbf{projecció} $\pi:T^*M\to M$ que envia $w\in T^*_pM$ a $p\in M$.


{\color{blue} Hi ha la propietat 3.18 del Lee que explica alguna cosa de que té dimensió $2n$, potser hauríem de mirar-ho.}
\subsection{Subvarietats, immersions, encabiments i difeomorfismes}
{\color{blue} Important motivar bé aquesta secció perquè és de les més importants per al nostre TFG}
\begin{defi}
    Sigui $\psi:M\to N$ una aplicació diferenciable {\color{blue} $C^\infty$}.
    \begin{enumerate}
        \item Diem que $\psi$ és una \textbf{immersió} si $d\psi_p$ és no-singular per a tot $p\in M$.
        \item Diem que $\psi$ és un \textbf{encabiment} (en anglès, \textit{embedding}) si és una immersió injectiva i un homeomorfisme sobre la seva imatge.
        \item Diem que $\psi$ és un \textbf{difeomorfisme} si és una injectiva amb inversa diferenciable {\color{blue} $C^\infty$}.
        \item El parell $(M, \psi)$ és una \textbf{subvarietat de $N$} si $\psi$ és una immersió injectiva.
    \end{enumerate}
\end{defi}

Un dels tipus de subvarietats que més ens interessarà seran les subvarietats encabides en $\mathbb R^n$. En concret, posarem particular atenció a $n=3$.

\begin{defi}
    Diem que $S\subseteq\mathbb R^3$ és una \textbf{superfície regular i simple} {\color{blue} no sé si cal di simple la veritat} si és la imatge d'un encabiment suau $\psi:T\to\mathbb R^3$ d'una regió elemental $T\subseteq\mathbb R^2$ en $\mathbb R^3$.
\end{defi}

\begin{ex}
    Siguin $R>r>0$. Per $T = [0,1)\times[0,1)\subseteq\mathbb R^2$, sigui $\psi:T\to\mathbb R^3$ l'aplicació
    \begin{equation*}
        \psi(u,v) = \left( (R+r\cos(2\pi v))\cos(2\pi u), (R+r\cos(2\pi v))\sin(2\pi u), r\sin(2\pi v) \right).
    \end{equation*}
    Aleshores, $\psi$ és un encabiment suau i $S = \psi(T)$ és una superfície regular i simple.
\end{ex}






\subsection{Camps vectorials}
\begin{defi}
    Sigui $M$ una varietat diferenciable suau, i $(a,b)$ un interval obert de $\mathbb R$. Una \textbf{corba suau} en $M$ és una aplicació diferenciable $\sigma:(a,b)\to M$. Si es pot estendre a un interval obert $(a-\epsilon,b+\epsilon)$ per algun $\epsilon>0$ escrivim també $\sigma:[a,b]\to M$. Definim el \textbf{vector tangent} a $\sigma$ en $t\in(a,b)$ com

    \begin{equation*}
        \dot{\sigma}_t = d\sigma\left(\frac{d}{dr}\Big|_{t}\right)\in T_{\sigma(t)}M
    \end{equation*}
\end{defi}

\begin{defi}
    Un \textbf{camp vectorial} $X$ \textbf{al llarg d'una corba} $\sigma:[a,b]\to M$ és una aplicació $X:[a,b]\to T(M)$ que \say{aixeca} $\sigma$, és a dir, tal que $\pi\circ X = \sigma$, on $\pi$ és la projecció del fibrat tangent tal com l'hem definit abans. 

    Un \textbf{camp vectorial} $X$ \textbf{en un conjunt obert} $U\subseteq M$ és una aplicació $X:U\to T(M)$ que \say{aixeca} $U$, és a dir, tal que $\pi\circ X = \text{id}_U$.
\end{defi}

En parlar de subvarietats encabides en espais euclidians $\mathbb R^n$, també serà interessant parlar del que anomenarem \textbf{vectors normals}.
\begin{defi}\label{def:espai_normal}
    Sigui $M\subseteq\mathbb R^n$ una subvarietat $m$-dimensional encabida en $\mathbb R^n$. Per tot punt $x\in M$, definim l'\textbf{espai normal} a $M$ en $x$, $N_xM$, com el subespai vectorial de $T_x\mathbb R^n$ ortogonal a $T_xM$ pel producte escalar euclidià.
\end{defi}



\subsection{Geometria Riemanniana}
Per tal de poder parlar de geometria en varietats diferenciables arbitràries, necessitem introduir el concepte de mètrica Riemanniana. Abans, però, recordem la definició de producte intern en $\mathbb R^n$.
\begin{defi}
    Un \textbf{producte intern} en un espai vectorial real $V$ és una aplicació $g:V\times V\to\mathbb R$ que és bilineal, simètrica i definida positiva, és a dir, que verifica
    \begin{enumerate}
        \item $g(\alpha u + \beta v, w) = \alpha g(u,w) + \beta g(v,w)$ per a tot $u,v,w\in V$ i $\alpha,\beta\in\mathbb R$,
        \item $g(u,v) = g(v,u)$ per a tot $u,v\in V$,
        \item $g(u,u)\ge0$ per a tot $u\in V$,
        \item $g(u,u) = 0$ si i només si $u = 0$.
    \end{enumerate}
\end{defi}

\begin{defi}
    Sigui $M$ una varietat diferenciable. Anomenem \textbf{mètrica Riemanniana} en $M$ una aplicació $g$ que assigna a cada punt $p$ en $M$ un producte intern $g_p:M_p\times M_p\to\mathbb R$ tal que per qualsevol obert $U\subseteq M$, si $X,Y$ són camps vectorials diferenciables en $U$, aleshores la funció $g(X,Y):U\to\mathbb R$ donada per
    \begin{equation*}
        g(X,Y)(p) = g_p(X_{|p},Y_{|p}),
    \end{equation*}
    és diferenciable.
    
    Anomenem \textbf{varietat Riemanniana} a una varietat diferenciable dotada d'una mètrica Riemanniana.
\end{defi}
En qualsevol sistema de coordenades tal que les coordenades siguin $z_1,\dots,z_n$, la mètrica Riemanniana es pot expressar com
\begin{equation*}
    g = \sum_{i,j=1}^n g_{ij}dz^i\otimes dz^j = g_{ij}dz^idz^j,
\end{equation*}
on $g_{ij}$ són funcions diferenciables en $U$.

\begin{defi}
    Anomenarem \textbf{mètrica euclidiana} en $\mathbb R^n$ la mètrica Riemanniana que en qualsevol sistema de coordenades tal que les coordenades siguin $z_1,\dots,z_n$ es pot expressar com
    \begin{equation*}
        \langle\cdot,\cdot\rangle = \delta_{ij}dz^idz^j,
    \end{equation*}
    on $\delta_{ij}$ és la delta de Kronecker.
\end{defi}

\begin{defi}
    Siguin $M$ i $N$ dues varietats Riemannianes, $g$ una mètrica Riemanniana en $N$ i $F:M\to N$ una aplicació diferenciable. Anomenem \textbf{pullback de $g$ per $F$} la mètrica Riemanniana $F^*g$ en $M$ tal que, per qualsevol parell de vectors tangents $u,v\in T_pM$,
    \begin{equation*}
        (F^*g)_p(v,u) = g_{F(p)}(dF_p(v),dF_p(u)),
    \end{equation*}
    on $dF_p$ és el diferencial de $F$ en $p$.
\end{defi}

\begin{prop}[\textbf{Existència de mètriques Riemannianes}]
    Tota varietat diferenciable suau admet una mètrica Riemanniana.
\end{prop}
{
    \color{green!50!black} \textit{Prova.}
    Sigui $M$ una varietat diferenciable suau {\color{blue}amb o sense frontera}. Sigui $\set{(U_\alpha,\varphi_\alpha)}$ un recobriment per cartes coordenades. En cada domini de carta, existeix una mètrica Riemanniana $g_\alpha = \varphi_\alpha^*\langle\cdot,\cdot\rangle = \delta_{ij}dz^idz^j$. Sigui $\set{\psi_\alpha}$ una partició de la unitat subordinada a $\set{U_\alpha}$. Definim 
    \begin{equation*}
        g = \sum_{\alpha} \psi_\alpha g_\alpha,
    \end{equation*}
    on els termes són zero fora dels suports de les $\psi_\alpha$. Com les particions de la unitat són localment finites, la suma és localment finita i per tant $g$ hereta la suavitat de les $g_\alpha$. És evidentment bilineal i simètrica per construcció, i només cal veure que és definida positiva.

    Sigui $v\in T_pM$ un vector tangent en $p\in M$ diferent de zero. Aleshores el producte intern definit en aquest punt és
    \begin{equation*}
        g_p(v,v) = \sum_\alpha \psi_\alpha(p) g_\alpha|_p(v,v)
    \end{equation*}
    que és una suma de termes no-negatius. Com a mínim alguna de les $\psi_\alpha$ és positiva en $p$ i, per tant, $g_p(v,v) > 0$.
    \qed
}

\begin{defi}
    Siguin $(M,g)$ i $(\tilde M,\tilde g)$ dues varietats Riemannianes {\color{blue} suaus}. Una aplicació {\color{blue} $C^\infty$} $F: M\to \tilde M$ és una \textbf{isometria (Riemanniana)} si és un difeomorfisme i $F^*\tilde g = g$. Diem que $F$ és una \textbf{isometria local} si tot punt $p\in M$ té un entorn $U$ tal que $F|_U$ és una isometria d'un entorn de $F(p)$ en $\tilde M$. Si $F$ és una isometria, diem que $M$ i $\tilde M$ són \textbf{isomètriques}, i si $F$ és una isometria local, diem que $M$ i $\tilde M$ són \textbf{localment isomètriques}.
\end{defi}

Hi ha diverses propietats de les varietats Riemannianes que són invariants per isometries. La més important per nosaltres és la planitud (en anglès, \textit{flatness}).
\begin{defi}\label{def:flatness}
    Diem que una varietat Riemanniana $(M,g)$ és \textbf{plana} si és isomètrica a l'espai euclidià $\mathbb R^n$ amb la mètrica euclidiana.
\end{defi}

{\color{blue} estaria molt bé demostrar que la planitud és invariant per isometries}
% \begin{teo}
%     Sigui $(M,g)$ una varietat Riemanniana. Aleshores, són equivalents:
%     \begin{enumerate}
%         \item $g$ és plana
%         \item Tot punt de $M$ està contingut en el domini d'una carta coordenada suau en què $g=\delta_{ij}dz^idz^j$
%         \item Tot punt de $M$ està contingut en el domini d'una carta coordenada suau en què el \textbf{coordinate frame} {\color{blue} Important mirar què és això} és ortonormal.
%         \item Tot punt de $M$ està contingut en el domini d'un \textbf{coordinate frame commutatiu} {\color{blue} Important mirar què és això}
%     \end{enumerate}
% \end{teo}
% {
%     \color{red} MIRAR DEMOSTRACIÓ!!!!
% }

\subsection{Un resultat interessant en $\mathbb R^3$}
{\color{blue} El que hi ha aquí s'ha de moure mes amunt}
{\color{red} EXPLICAR SUBVARIETATS RIEMANNIANES COM EN EL LEE, DE MANERA QUE TENIM LA MÈTRICA INDUÏDA. A MÉS; EN AQUEST CAPÏTOL ES PARLA DEL TEOREMA DE WHITNEY CAL DEFINIR SUPERFÍCIE REGULAR EN R3 I DEMOSTRAR QUE EL TOR ÉS SUPERFÍCIE REGULAR COMPACTA. CAL DEFINIR CURVATURA DE GAUSS I ENUNCIAR I DEMOSTRAR EL TEOREMA QUE TOTA SUPERFÍCIE REGULAR COMPACTA TÉ COM A MÍNIM UN PUNT ELÍPTIC. POTSER TAMBÉ CAL EXPLICAR SEGONA FORMA FONAMENTAL UF}

Si $(M,g)$ és una varietat Riemanniana, qualsevol subvarietat diferenciable $S\subseteq M$ admet una \textbf{mètrica induïda} $\imath^*g$, on $\imath:S\hookrightarrow M$ és la inclusió.

\begin{defi}
    Sigui $M$ una varietat diferenciable suau. {\color{blue} Això està ben dit?}. Una \textbf{subvarietat encabida} (en anglès, \textit{embedded submanifold}) de $M$ és un subconjunt $S\subseteq M$ que és una varietat topològica en la topologia induïda, dotada d'una estructura suau{\color{blue} això ho he definit?} tal que la inclusió $\imath:S\to M$ és un encabiment suau. 
\end{defi}

{\color{blue} Cal demostrar que el tor és superfície regular compacta}
En el cas de superfícies regulars en $\mathbb R^3$, l'espai normal tal com l'hem definit a la definició \ref{def:espai_normal} en qualsevol punt de la superfície és unidimensional. En general, el que ens interessarà d'aquest espai normal és la seva direcció, de manera que definim l'aplicació següent:
\begin{defi}
    Sigui $S$ una superfície regular. Anomenem \textbf{aplicació de Gauss} o \textbf{aplicació normal} de $S$ a una aplicació $N:S\to \mathbb S^2$ que a cada punt $p\in S$ li assigna un vector normal unitari a $S$ en $p$.
\end{defi}
Diem que $S$ és \textbf{orientable} si existeix una aplicació de Gauss $N$.
\begin{ex}
    La cinta de Möbius, donada per $S=\phi(\mathbb R \times (-1,1))$, on $\phi(u,v) = 2(\cos(2u), \sin(2u), 0) + v(\cos(u)\cos(2u), \cos(u)\sin(2u), \sin(u))$, és una superfície regular no orientable.
\end{ex}
Un dels resultats més rellevants pel que fa a l'orientació de superfícies regulars, que no demostrarem aquí, és el següent:
\begin{teo}
    Tota superfície regular compacta és orientable.
\end{teo}

El diferencial de l'aplicació de Gauss $N$ d'una superfície regular orientable $S$, $dN_p:T_pS\to T_{N(p)}\mathbb S^2$, es pot interpretar com un operador lineal del pla tangent a $S$ en si mateix, ja que el pla tangent a $\mathbb S^2$ en $N(p)$ és el mateix que el pla tangent a $S$ en $p$. Per veure això, només cal considerar que $T_{N(p)}\mathbb S^2 = N(p)^\perp = T_pS$.{\color{blue} això és una mica handwavey} Per aquest motiu, denotem \textbf{endomorfisme de Weingarten} $W_p$ el diferencial de l'aplicació de Gauss quan és considerat com un endomorfisme de $T_pS$. Amb aquest endomorfisme, podem arribar a definir la curvatura d'una superfície regular orientada.

\begin{defi}
    Sigui $S$ una superfície regular orientada, $p$ un punt de $S$. Anomenem \textbf{segona forma fonamental} de $S$ en $p$ la forma bilineal
    \begin{align*}
        II_p:T_pS\times T_pS&\to\mathbb R\\
        (v,w)&\mapsto \langle W_p(v),w\rangle
    \end{align*}
    on $\langle\cdot,\cdot\rangle$ és el producte escalar euclidià en $T_pS$.
\end{defi}

\begin{defi}
    Sigui $S$ una superfície regular orientada, $p$ un punt de $S$. Anomenem \textbf{curvatura de Gauss} $\kappa_S(p)$ de $S$ en $p$ el determinant de l'endomorfisme de Weingarten $W_p$.
\end{defi}

Es pot demostrar, {\color{blue} potser ho hauriem de fer, esta als apunts de l'Ignasi de geodif} de fet, que la l'endomorfisme de Weingarten diagonalitza amb valors propis reals $k_1$ i $k_2$, de manera que la curvatura de Gauss és
\begin{equation*}
    \kappa_S(p) = k_1(p)k_2(p).
\end{equation*}
Anomenem $k_1(p)$ i $k_2(p)$ les \textbf{curvatures principals} de $S$ en $p$, i els vectors propis corresponents a aquests valors propis s'anomenen \textbf{direccions principals de curvatura}. Localment, corbes sobre la superfície que passen per $p$ i segueixen les direccions principals de curvatura coincideixen amb cercles encabits a $\mathbb R^3$ tangents a aquest mateix punt i de radis $1/|k_1(p)|$ i $1/|k_2(p)|$.  La curvatura de Gauss és negativa quan els centres d'aquests cercles, que anomenem \textbf{centres de curvatura}, es troben en costats oposats del pla tangent, i positiva quan estan en el mateix costat.  {\color{blue}mirar si això està ben explicat.}

\begin{defi}
    Sigui $S$ una superfície regular orientada, $p$ un punt de $S$. Diem que $p$ és un \textbf{punt el·líptic} si $\kappa_S(p) > 0$.
\end{defi}

Intuïtivament, una superfície regular és el·líptica en un punt si, localment, la superfície roman a un mateix costat del pla tangent, sense creuar-lo. 

\begin{teo}
    Tota superfície regular compacta té un punt el·líptic.
\end{teo}
{
    \color{green!50!black}
    \textit{Prova.}
    Sigui $S$ una superfície regular compacta, i considerem l'aplicació 
    \begin{align*}
        g:S&\to\mathbb R\\
        p&\mapsto ||p||^2.
    \end{align*}
    Com $S$ és compacta i $g$ és una funció real contínua, $g$ assoleix un màxim en un punt $p_{\max}\in S$. Sigui $R^2 = ||p_{\max}||^2$ el valor màxim de $g$. 
    
    Com a $p_{\max}$ s'assoleix el màxim de la distància a l'origen, el vector posició $p_{\max}$ és ortogonal a la superfície en $p_{\max}$. És a dir, $p_{\max}$ és un punt comú de la superfície regular $S$ i la frontera de l'esfera $B_{R}(0)$, on els espais tangents a $S$ i $B_{R}(0)$ coincideixen. {\color{blue} crec que això és el que l'ignasi va quedar-se demostrant quan ho vam fer a classe}

    Tots els punts de $S$ tenen distància a l'origen menor que $R$, de manera que $S\subseteq B_{R}(0)$. Siguin $k_1$ i $k_2$ les curvatures principals de $S$ en $p_{\max}$, i $\alpha:[0,1]\to S$ i $\beta:[0,1]\to S$ corbes tangents a $S$ en $p_{\max}$ que corresponen a les direccions principals de curvatura. Les curvatures principals de l'esfera $B_{R}(0)$ en $p_{\max}$ són ambdues $1/R$ o ambdues $-1/R$, depenent de la orientació $N$ escollida. Escollim una orientació $N$ de $S$ tal que en $p_{\max}$ les curvatures principals de l'esfera siguin $1/R$. Aleshores, si alguna de les curvatures principals de $S$ és menor que $1/R$, la corba $\alpha$ o $\beta$ tindrà un radi de curvatura més gran que el de l'esfera, de manera que hi haurà un punt $\tilde p\in S$ proper a $p_{\max}$ que té distància a l'origen més gran que $R$. Això entra en contradicció amb el fet que la distància a l'origen de $p_{\max}$ és màxima. Per tant, ambdues curvatures principals de $S$ en $p_{\max}$ han de ser majors que $1/R$, i per tant $S$ té un punt el·líptic.
    \qed
}

A continuació enunciem un dels teoremes més importants de la geometria de corbes i superfícies regulars en $\mathbb R^3$, demostrat per Carl Friedrich Gauss el 1827, que relaciona la curvatura de Gauss d'una superfície regular amb la seva mètrica com a varietat Riemanniana.

\begin{teo}[Teorema Egregi de Gauss]
    La curvatura de Gauss d'una superfície regular només depèn de la mètrica de la superfície com a varietat Riemanniana. En concret, la curvatura de Gauss és invariant per isometries. 
\end{teo}

Un cas particular d'aquest teorema és que la planitud tal com l'hem definit a la definició \ref{def:flatness}, que la mètrica d'una varietat Riemanniana sigui la mètrica euclidiana, és equivalent a que la curvatura de Gauss sigui nul·la. Això té una implicació a l'hora de determinar quines varietats Riemannianes bidimensionals es poden encabir de manera isomètrica en $\mathbb R^3$. En concret, obtenim el següent resultat per al tor pla. 

\begin{teo}
    Anomenem \textbf{tor pla} la varietat Riemanniana $(\mathbb T^2, g)$ on $\mathbb T^2 = \mathbb R^2/\mathbb Z^2$ és el tor, i $g$ és la mètrica euclidiana. No existeix cap encabiment isomètric $C^\infty$ del tor pla en $\mathbb R^3$.
\end{teo}



\newpage