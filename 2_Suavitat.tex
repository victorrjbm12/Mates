\chapter{Suavitat i diferenciabilitat}
{\color{blue} L'Ignasi m'ha dit que el que hauria de ressaltar més és la diferència entre $C^1$ i $C^\infty$. Potser podria comentar alguns dels resultats que són vàlids en un i no en l'altre?}
\section{Suavitat}
{\color{blue} El que hi ha aquí ho estic traient de wikipedia. Preguntar a l'Ignasi per una font més bona.}
\begin{defi}
    Sigui $U\subseteq\mathbb R$ un conjunt obert, $f:U\to\mathbb R$ una funció real contínua.
    Diem que $f$ és {\normalfont $k$-vegades derivable contínuament}, amb $k\in\mathbb N_0$, si la derivada d'ordre $k$, $$f^{(k)}:= \frac{d^k}{dx^k}f,$$ existeix i és contínua en $U$. Anomenem l'índex $k$ \textit{suavitat} de $f$.
    Diem que $f$ és {\normalfont suau} o {\normalfont infinitament derivable} si existeix la derivada de qualsevol ordre.
\end{defi}
\begin{obss}
\end{obss}
\begin{itemize}
    \item $f$ és $0$ vegades derivable contínuament si i només si $f$ és contínua. 
    \item Si $f$ és $k$ vegades derivable contínuament, aleshores també és $j$ vegades derivable contínuament per $0\le j\le k$.
\end{itemize}
\begin{defi}
    Anomenem \textit{classe de diferenciabilitat} $C^k(U)$, amb $k\in\mathbb N_0$ l'espai de les funcions $k$-vegades derivables contínuament en $U\subseteq\mathbb R^n$. Anomenem $C^\infty(U)$ l'espai de les funcions infinitament derivables en $U$.
\end{defi}
De la mateixa manera, podem definir aquests conceptes per funcions de diverses variables.
\begin{defi}
    Sigui $U\subseteq\mathbb R^n$ un conjunt obert, $f:U\to\mathbb R$ una funció real contínua.
    Diem que $f$ és \textit{$k$-vegades derivable contínuament}, amb $k\in\mathbb N_0$, si totes les seves derivades parcials d'ordre $k$, $$\frac{\partial^k}{\partial x_1^{\alpha_1}\cdots\partial x_n^{\alpha_n}}f,$$ tals que $\sum_{i=1}^n\alpha_i = k$, existeixen i són contínues en $U$. És a dir, si $f$ és $k$-vegades derivable contínuament en cada component de $U$.
\end{defi}
\begin{defi}
    Anomenem \textit{classe de diferenciabilitat} $C^k(U)$, amb $k\in\mathbb N_0$ l'espai de les funcions $k$-vegades derivables contínuament en $U\subseteq\mathbb R^n$. Anomenem $C^\infty(U)$ l'espai de les funcions infinitament derivables en $U$.
\end{defi}
\begin{obss}{\color{blue} Potser caldria donar la definició explícita en el cas de funcions de diverses variables.}
\end{obss}
\begin{itemize}
    \item Moltes vegades, en lloc de $C^k(U)$ es fa servir $C^k(U;\mathbb R^m)$ per a funcions $f:U\to\mathbb R^m$.
    \item També escriurem simplement $C^k$ en lloc de $C^k(U)$ si el domini és clar pel context.
\end{itemize}

\subsection{Classes de diferenciabilitat com espais normats}
A continuació definirem normes en $C^k(U)$ i $C^\infty(U)$ que ens permetran tractar-les com espais vectorials normats.
\begin{defi}
    Sigui $U\subseteq\mathbb R^n$ un conjunt obert, $f:U\to\mathbb R^m$ una funció. Definim la norma $\|\cdot\|_{C^k(U)}$ de $f$ com
    \begin{equation*}
        \|f\|_{C^k(U)} = \sum_{|\alpha|\le k} \|D^\alpha f\|_{L^\infty(U)},
    \end{equation*}
    on $\|\cdot\|_{L^\infty(U)}$ és la norma del suprem en $U$, tal que $\|g\|_{L^\infty(U)} = \sup_{x\in U} |g(x)|$.
\end{defi}
{\color{blue} Aquí podríem proposar i demostrar que $\|\cdot\|_{C^k(U)}$ és efectivament una norma, ho tenim a \url{https://proofwiki.org/wiki/C%5Ek_Norm_is_Norm}. També cal decidir si definim explícitament la $D^\alpha$.}

\section{Fonaments de geometria diferencial}
Aquí hauriem de parlar d'embedding preferiblement.






\newpage
