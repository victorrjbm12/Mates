% chktex-file 1
\documentclass[11pt,a4paper,openright,oneside]{book}
\usepackage{amsfonts, amsmath, amssymb,latexsym,amsthm, mathrsfs, enumerate}
\usepackage[catalan]{babel}
\usepackage{epsfig}

\usepackage{mathtools}
\usepackage{textcomp}
\usepackage{gensymb}
\usepackage{natbib}
\usepackage{comment}
\usepackage{microtype}
\usepackage{multicol}
\usepackage{amsfonts}
\usepackage{bbm}

%\usepackage[utf8]{inputenc}
\usepackage{pgfplots}
\usepackage{tikz}
\usetikzlibrary{calc}
\usetikzlibrary{arrows}
\usepackage{wrapfig}
\usepackage{fancyhdr}
\usepackage{parskip}

\parskip=5pt
\parindent=15pt
\usepackage[margin=1.2in]{geometry}
\usepackage{graphicx}
\usepackage{listings}
\usepackage[utf8]{inputenc}
\usepackage[T1]{fontenc}

\DeclarePairedDelimiter{\ceil}{\lceil}{\rceil}
\DeclarePairedDelimiter{\set}{\{}{\}}

\setcounter{page}{0}


\numberwithin{equation}{section}
\newtheorem{teo}{Teorema}[section]
\newtheorem*{teo*}{Teorema}
\newtheorem*{prop*}{Proposici\'o}
\newtheorem*{corol*}{Coro{\l}ari}
\newtheorem{prop}[teo]{Proposici\'o}
\newtheorem{corol}[teo]{Coro{\l}ari}
\newtheorem{lema}[teo]{Lema}
\newtheorem{defi}[teo]{Definici\'o}
\newtheorem{nota}{Notaci\'o}


\theoremstyle{definition}
\newtheorem{prob}[teo]{Problema}
\newtheorem*{sol}{Soluci\'o}
\newtheorem{ex}[teo]{Exemple}
\newtheorem{exs}[teo]{Exemples}
\newtheorem{obs}[teo]{Observaci\'o}
\newtheorem{obss}[teo]{Observacions}

\def\qed{\hfill $\square$}

\renewcommand{\refname}{Bibliografia}
% --------------------------------------------------
\usepackage{fancyhdr}

\lhead{}
\lfoot{}
\rhead{}
\cfoot{}
\rfoot{\thepage}

\pgfplotsset{compat=1.18}

% Add these accent definitions
\DeclareUnicodeCharacter{00ED}{\'i}
\DeclareUnicodeCharacter{00E9}{\'e}
\DeclareUnicodeCharacter{00E0}{\`a}
\DeclareUnicodeCharacter{00E8}{\`e}
\DeclareUnicodeCharacter{00F3}{\'o}
\DeclareUnicodeCharacter{00FA}{\'u}

\begin{document}

\bibstyle{plainnat}

\thispagestyle{empty}

\begin{titlepage}
\begin{center}
\begin{figure}[htb]
\begin{center}
\includegraphics[width=6cm]{matematiquesinformatica-pos-rgb.png}
\end{center}
\end{figure}

\vspace*{1cm}
\textbf{\LARGE GRAU DE MATEM\`{A}TIQUES } \\
\vspace*{.5cm}
\textbf{\LARGE Treball final de grau} \\

\vspace*{1.5cm}
\rule{16cm}{0.1mm}\\
\begin{Huge}
\textbf{EL MEU TFG} \\
\end{Huge}
\rule{16cm}{0.1mm}\\

\vspace{1cm}

\begin{flushright}
\textbf{\LARGE Autor: Víctor Rubio Jiménez}

\vspace*{2cm}

\renewcommand{\arraystretch}{1.5}
\begin{tabular}{ll}
\textbf{\Large Director:} & \textbf{\Large Dr. Ignasi Mundet i Riera } \\
\textbf{\Large Realitzat a:} & \textbf{\Large  Departament de matemàtiques i informàtica   } \\
\\
\textbf{\Large Barcelona,} & \textbf{\Large \today }
\end{tabular}

\end{flushright}

\end{center}










\end{titlepage}


\newpage
\pagenumbering{roman} 

\section*{Abstract}
My wonderful abstract.

\section*{Resum}
\begin{itemize}
    \item Explicar diferència entre Continu i Diferenciable i totes les restriccions
    \item Explicar C1 i mobius
    \item Mirar tor ambient space isometric embeddings square flat torus
    \item MIrar article Nash, Gromov
    \item Mirar quin Cn agafa al paper band
\end{itemize}

\newpage 


\section*{Agra\"{\i}ments}

Bla, bla, bla
\newpage

\tableofcontents

\newpage

\chapter{Introducci\'o}



\subsection*{Objectius del treball}

\begin{itemize}
    \item Explicats
\end{itemize}

\subsection*{Estructura de la mem\`oria}
Tremendo

\subsection*{Guia de lectura}
Faig servir incrustació, que potser hauria de dir immersió?

\newpage

\pagenumbering{arabic}
\setcounter{page}{1}

\chapter{Comencem}


\newpage
\chapter{Teorema Nash-Kuiper C1}
\section{Explicació del Sung-Jin Oh}
\begin{teo}\label{teo: SJO} Sigui $(M,g)$ una superfície, $N\ge\dim M+1$ i $u:M\to\mathbb R^N$ una incrustació estrictament curta, és a dir, tal que la longitud de cada vector en $M$ s'escurça (estrictament) sota $\nabla u$. Aleshores $u$ es pot aproximar uniformement per incrustacions isomètriques $C^1$.
\end{teo}
{\color{blue} Haurem d'explicar què és una incrustació isomètrica, i mirar si cal que li canviem el nom}

Per exemple, l'homotècia $\mathbb S^2\to\varepsilon\mathbb S^2$ amb $\varepsilon\in(0,1)$ és una aplicació \textit{curta}. 
\begin{obs}
De fet, qualsevol incrustació $C^2$ isomètrica $u:\mathbb S^2\hookrightarrow\mathbb R^3$ ha de ser igual a la incrustació estàndard $\mathbb S^2 \hookrightarrow\set{ x\in\mathbb R^3 : |X| = 1 }$ fins translació i rotació. 
\end{obs}

El teorema \ref{teo: SJO} es demostra amb \textit{integració convexa}.

Sung-Jin Oh demostra aquí el Baby Nash theorem.


Sigui $D=\set{x\in\mathbb R^2 : |x|<1}$ el disc unitat, i $g = g_{ij}(x)$ una mètrica de $D$. Una aplicació $u:D\to\mathbb R^n$ és una \textit{immersió} {\color{blue} per ara direm incrustació als embeddings i immersions als \textit{immersions}} si $\nabla u(x)$ és injectiva per tot $x$. La mètrica en $D$ induïda per $u$ és de la forma {\color{blue} Haurem de recordar i controlar el que era la mètrica induïda}
\begin{equation*}
    \nabla u ^{\intercal}(x)\nabla u (x) = \begin{pmatrix}
    \nabla_1 u\nabla_1 u & \nabla_1 u\nabla_2 u\\
    \nabla_2 u\nabla_1 u & \nabla_2 u\nabla_2 u
    \end{pmatrix}
\end{equation*}
Diem que l'aplicació és \textit{isomètrica} {\color{blue} també estaria bé comentar isometries} si $\nabla u ^{\intercal}\nabla u = g$, i diem que és \textit{(estrictament) curta} si $\nabla u(x)^{\intercal} \nabla u(x) - g(x) \le 0$ per tot $x\in D$.
\begin{teo}
    Sigui $n\ge 4$ i $u:D\to\mathbb R^n$ una immersió isomètrica estrictament curta. Per qualsevol $\varepsilon > 0$, existeix una immersió isomètrica $C^1$ $\tilde u:D\to\mathbb R^n$ tal que $\|u-v\|_{C^0(D)} < \varepsilon$.{\color{blue} Aquí cal motivar l'interès d'aquest resultat. Entenc que la idea és que partim d'una immersió que és estrictament curta i volem trobar una que sigui contínua i isomètrica. Vull mirar continuïtats.}
\end{teo}
\begin{obs}
    Aquest teorema necessita que la co-dimensió sigui com a mínim 2.
\end{obs}
La manera de demostrar aquest resultat és a través d'un mètode iteratiu amb passos altament oscil·lants.
Sigui $u_1 = u + U$ amb 
\begin{equation*}
    U=\sum_{I\in\mathcal I} U_I
\end{equation*}
Volem que cada component $U_I^j$ sigui complex, per tal que oscil·li com $e^{ix \cdot \xi}$, però que el resultat del sumatori sigui real. Així, imposem per cada $I\in\mathcal I$ que existeixi $\overline I\in\mathcal I$ tal que 
\begin{equation*}
    U_{\overline{I}} = \overline{U}_I, \quad \overline{\overline{I}} = I.
\end{equation*}
Ara tenim un error mètric $h_1 = g - \nabla u_1 ^{\intercal}\nabla u_1$.
\begin{equation*}
    h_1 = 
        \left( h-\sum_{I} \nabla \overline{U}_I ^{\intercal}\nabla U_I \right) 
        - \sum_{I}\left( \nabla u ^{\intercal}\nabla U_I + \nabla U_I ^{\intercal}\nabla u \right)
        - \sum_{I,J: J\not = \overline I} \nabla U_I ^{\intercal}\nabla U_J.
\end{equation*}
I anomenem els tres sumands, en ordre, $q_{\text{mèt}}$, $q_{\text{lin}}$ i $q_{\text{alt}}$.

Volem una correcció que oscil·li en una sola direcció $\xi\in\mathbb R^2$, $|\xi|=1$. Posem
\begin{equation*}
    U_I = W = \frac1\lambda a(x)\textbf{n}(x)e^{\lambda i x \cdot \xi},
\end{equation*}
amb $a:D\to\mathbb R$ i $\textbf{n}:D\to\mathbb C^n$ tal que $\textbf{n}\cdot \overline{\textbf{n}}=1$. Per tal que sigui real, definim també $\overline I\in\mathcal I$ tal que 
\begin{equation*}
    U_{\overline{I}} = \overline W = \frac1\lambda a(x)\overline{\textbf{n}}(x)e^{-\lambda i x \cdot \xi}.
\end{equation*}

Per eliminar el terme $q_{\text{mèt}}$, observem que
\begin{equation*}
    \begin{aligned}
    \nabla_j W &= i\xi_j a(x)\textbf{n}(x)e^{\lambda i x \cdot \xi} + \frac{1}{\lambda}\nabla_j (a(x)\textbf{n}(x)e^{\lambda i x \cdot \xi})\\
    &=i\xi_j a(x)\textbf{n}(x)e^{\lambda i x \cdot \xi} + O(\frac{1}{\lambda})
    \end{aligned}
\end{equation*}
EXPLICAR PER QUÈ ÉS O(1/LAMBDA)!!!
I, per tant,
\begin{equation*}
    \begin{aligned}
    \nabla_i W^*(x) \nabla_jW(x)&= (-i\xi_ia(x)e^{-\lambda i x \cdot \xi})(i\xi_ja(x)e^{\lambda i x \cdot \xi})\overline{\textbf{n}}(x) \cdot\textbf{n}(x) + O(\frac{1}{\lambda})\\
    &= \xi_i\xi_j a(x)^2 + O(\frac{1}{\lambda})
    \end{aligned}
\end{equation*}
on definim $(\cdot)^* = (\overline{\cdot})^\intercal$. Així, l'oscil·lació és cancel·lada i en resulta un terme $a(x)^2\xi_i\xi_j$.
\begin{ex}
    EXPLICAR MILLOR AQUEST EXEMPLE
    Posem que per un cert $x\in D$, l'error $h$ és de la forma
    \begin{equation*}
        h(x) = a^2(x)\xi \otimes \xi + b^2(x)\xi' \otimes \xi' + c^2(x)\xi'' \otimes \xi''
    \end{equation*}
    aleshores, amb això fem desaparèixer el terme $\xi \otimes \xi$. Repetint-ho per $\xi' \otimes \xi'$ i $\xi'' \otimes \xi''$ aconseguim reduir l'error $h(x)$ a un terme $O(\frac{1}{\lambda})$.
\end{ex}
\begin{obs}
    EXPLICAR AQUESTA OBSERVACIÓ
    Aquest mètode requereix que $h$ sigui curta, ja que $\nabla_i W^*(x) \nabla_jW(x)$ és un terme no-negatiu. De fet, per tal que $h_1$ sigui curt, necessitem que $h$ sigui estrictament curt.
\end{obs}
Ara bé, els autovectors $\xi$ depenen d'$x$. Això es pot resoldre amb el següent lema.
\begin{lema} (Descomposició de l'error mètric)
    Sigui $\mathcal P$ l'espai de totes les matrius definides positives. Existeix una successió $\xi^{(k)}$ de vectors unitaris en $\mathbb R^n$ i una successió $\Gamma_{(k)}\in C_c^\infty(\mathcal P; [0,\infty))$ tals que
    \begin{equation*}
        A_{ij} = \sum_k\Gamma^2_{k}(A)\xi_i^{(k)}\xi_j^{(k)}
    \end{equation*}
    i aquesta suma és \textit{localment finita}. És a dir, existeix $N\in\mathbb N$ tal que per tot $A\in\mathcal P$ com a màxim $N$ termes de $\Gamma_{(k)}$ són no-nuls.
\end{lema}
\begin{obs}
    La demostració d'aquest teorema no l'escrivim aquí explícitament. ESTÀ AL SUNG-JIN OH.
\end{obs}
Fins ara no ha calgut especificar el vector $\textbf{n}(x)\in\mathbb C^n$ per tal de minimitzar l'error mètric. Veurem que el podem escollir de tal manera que els termes $q_{\text{lin}}$ i $q_{\text{alt}}$ desapareguin fins a terme $O(1/\lambda)$.
\begin{itemize}

    \item \textbf{Error de linearització.} Substituïm el terme amb $W$
    \begin{equation*}
         \nabla_i u ^{\intercal}\nabla_j W = i\xi_j a(x)e^{ix\cdot\xi}\nabla_i u\cdot\textbf{n}+O(1/\lambda)
    \end{equation*}
    i veiem que podem eliminar aquest component escollint un vector perpendicular a l'espai tangent de $u(x)$, $\textbf{n}(x)\perp \nabla_j u(x)$. Això es pot fer perquè l'espai té co-dimensió 1 (REVISAR!!!)Podem fer el mateix amb $\nabla_iW^{\intercal}\nabla_ju$ i obtenim
    \begin{equation*}
        \nabla_i u ^{\intercal}\nabla_j W + \nabla_iW^{\intercal}\nabla_ju = O(1/\lambda)
    \end{equation*}

    \item \textbf{Interferència altament oscil·lant.} De nou, substituïm el terme
    \begin{equation*}
        \nabla_iW^{\intercal}\nabla_jW = (-a^2(x)\xi_i\xi_je^{2ix\cdot\xi})\textbf{n}\cdot\textbf{n} + O(1/\lambda).
    \end{equation*}
    I ara només cal utilitzar que la incrustació té co-dimensió $\ge2$ per escollir un vector complex tal que $\textbf{n}\cdot\textbf{n} = 0$. Podem prendre, per exemple, 
    \begin{equation*}
        \textbf{n} = \frac{1}{i\sqrt2}\zeta(x) + \frac{1}{\sqrt2}\eta(x)
    \end{equation*}
    on $\zeta(x)$ i $\eta(x)$ són vectors reals unitaris ortogonals a l'espai tangent $T_{u(x)}u(D)$.
    \item \textbf{Forma final de la correcció.} Tot plegat, tenim una correcció de la forma
    \begin{equation*}
        W(x) = \frac{a(x)}{\lambda}\left( \sin(\lambda x \cdot \xi)\zeta(x) + \cos(\lambda x \cdot \xi)\eta(x) \right)
    \end{equation*}
    amb les següents propietats:
    \begin{itemize}
        \item[--] Norma $C^0$ petita: $$||W||_{C^0}\le C\frac{||a||_{C^0}}{\lambda}$$
        \item[--] Terme principal en $\nabla W$ $$\nabla W = a(x)\left( \cos(\lambda x \cdot \xi)\zeta(x) - \sin(\lambda x \cdot \xi)\eta(x) \right) + O_{||a||_{C^0}, ||\nabla a||_{C^0}, ||\nabla\zeta||_{C^0}, ||\nabla\eta||_{C^0}}(1/\lambda)$$
        \item[--] Error mètric petit: $$\nabla_i W^{\intercal}\nabla_j W(x) -a^2(x)\xi_i\xi_j = O(1/\lambda)$$
        \item[--] Error de linearització petit: $$\nabla_i u ^{\intercal}\nabla_j W + \nabla_iW^{\intercal}\nabla_ju = O(1/\lambda)$$
        \item[--] Error d'interferència petita: $$\nabla_i W ^{\intercal}\nabla_j W = O(1/\lambda)$$
    \end{itemize}
\end{itemize}
\begin{obs}
    AIXÒ S'HA DE MIRAR BÉ, NO ACABO D'ENTENDRE COM FA LA DERIVADA
    Una manera alternativa d'arribar a la forma general de la correcció és la següent. Definim 
    \begin{equation*}
        \begin{aligned}
        \gamma = (\gamma_1, \gamma_2) : &D\times \mathbb T\to \mathbb R^2\\
        & (x,t)\mapsto \gamma(x,t)
        \end{aligned}
    \end{equation*}
    on $\mathbb T = \mathbb R / 2\pi\mathbb Z$.
    Posant $\dot\gamma$ la derivada respecte de $t$, tenim que
    \begin{equation*}
        \nabla W ^{\intercal}(x) \nabla W(x) = \left( \dot\gamma_1^2(x, \lambda x\cdot\xi) + \dot\gamma_2^2(x, \lambda x\cdot\xi) \right)\xi\otimes\xi + O(1/\lambda).
    \end{equation*}
    De manera que per cada $x$ cal trobar $\gamma(x,\cdot)$ tal que 
    (1) $\dot\gamma_1^2 + \dot\gamma_2^2 = a^2$ i 
    (2) $t\mapsto \dot\gamma(x,t)$ sigui $2\pi$-periòdic i $\int\dot\gamma\text{d}t = 0$
    De manera que $t\mapsto\gamma(x,t)$ també ha de ser $2\pi$-periòdica i el seu origen ha de pertànyer al disc unitat tancat $\overline D$.
    IMPORTANT Per a després, NASH ANOMENA STEP A CADA ADDICIÓ D'UNA CORRECCIÓ, que es carrega un terme a un error d'ordre $O(1/\lambda)$. 
\end{obs}
\begin{lema}
    Sigui $u:D\to\mathbb R^n$ una immersió suau estrictament curta, tal que $h:=g-\nabla u ^{\intercal}\nabla u$ obeeix
    \begin{equation*}
        ||h||_{C^0} \le e_h
    \end{equation*}
    per algun $e_h > 0$. Aleshores, per qualsevol $\varepsilon > 0$, existeix una immersió suau estrictament curta $u_{[1]} = u + U$, on
    \begin{equation*}
        \begin{aligned}
        ||U||_{C^0(D)} &\le \varepsilon\\
        ||\nabla U||_{C^0(D)} &\le Ce_h^{1/2}
        \end{aligned}
    \end{equation*}
    i $h_{[1]}:=g-\nabla u_{[1]}^{\intercal}\nabla u_{[1]}$ obeeix
    \begin{equation*}
        ||h_{[1]}-h||_{C^0} \le \varepsilon.
    \end{equation*}
\end{lema}




\newpage
\chapter{Paper Mobius strip}

\newpage
\chapter{Conclusions}

Hem après un munt

\normalfont

\newpage

\begin{thebibliography}{25}

\bibitem[Autor1 \& Autor2(ANY)]{nomdelacita}
Autor1, A., \& Autor2, B. (ANY).
\newblock \textit{Nom del treball.}
\newblock Cambridge University Press.

\end{thebibliography}
\end{document} 

