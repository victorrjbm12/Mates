\chapter{La cinta de Möbius de paper òptima}

Al capítol anterior hem vist que, tot i que pel teorema de Whitney sempre existeix algun espai euclidià de dimensió prou alta tal que s'hi pugui encabir suaument una varietat diferenciable $C^\infty$, no sempre és possible encabir-los en un espai de dimensió 3. A més, fins i tot per varietats riemannianes que sí es poden encabir suaument en $\mathbb R^3$, hem vist que no sempre és possible encabir-les isomètricament, com és el cas del tor pla $(\mathbb T^2, g)$. 

A continuació presentem un resultat obtingut recentment, l'any 2023, per Richard Evan Schwartz, on demostra una altra restricció per una varietat amb frontera, aquesta vegada en el cas de les cintes de Möbius planes rectangulars. Veurem que, tot i que és possible encabir-les suaument i isomètrica en $\mathbb R^3$ quan el costat llarg de la cinta de Möbius és $\sqrt3$ vegades major que el curt, no és possible fer-ho quan és més petit. Parlarem de cintes de Möbius \say{de paper} degut a la familiaritat que es té amb aquest tipus d'objectes i pel fet que qualsevol manipulació suau d'un tros de paper és un encabiment isomètric, per la poca flexibilitat del material.

L'objectiu d'aquest capítol no serà només veure aquest resultat, sinó també veure com el podem \say{esquivar} relaxant les condicions del nostre encabiment isomètric. La manera en què ho aconseguirem apuntarà cap al que veurem al capítol següent, on demostrarem el teorema de Nash. 

La font d'aquest capítol és gairebé exclusivament la versió més recent de \citet{schwartz2024}. Aquest article és molt senzill, curt i fàcil de llegir, de manera que seguim de manera molt propera la seva demostració. Hem afegit els diagrames de la demostració del lema T per tal de fer més evident l'argument topològic que fa servir Schwartz.


\section{Introducció al problema}
Comencem introduint un tipus d'objecte que no havíem definit fins ara, la varietat topològica amb frontera. No dedicarem gaire espai a estudiar-les per separat, però és important entendre que molts dels conceptes i resultats que hem vist al capítol anterior es poden generalitzar a varietats topològiques amb frontera, tractant la frontera com un altra varietat topològica de dimensió $n-1$.
\begin{defi}\label{def:varietat_topologica_amb_frontera}
    Sigui $M$ un espai topològic Hausdorff i tal que verifica el segon axioma de numerabilitat. Diem que $M$ és una \textbf{varietat topològica amb frontera de dimensió $n$} si per a cada $p\in M$ existeix un entorn obert $U\subseteq M$ de $p$ que és homeomorf a un obert de $\mathbb R^n$ o a un obert del semiespai superior $\mathbb H^n := \set{(x_1, \dots, x_n)\in\mathbb R^n:x_n\geq 0}$. 
\end{defi}

\begin{nota}\label{nota:frontier_and_interior}
    Escriurem $\partial M$ per denotar \textbf{la frontera de $M$}, és a dir, el subconjunt de punts de $M$ al voltant dels quals la varietat topològica amb frontera té un entorn homeomorf a $\mathbb H^n$ però no en té cap d'homeomorf a $\mathbb R^n$. Recíprocament, anomenarem $int(M)$ l' \textbf{interior de $M$}, és a dir, els punts de $M$ que no pertanyen a $\partial M$.
\end{nota}

A continuació, presentem la definició de la cinta de Möbius plana rectangular i del seu encabiment isomètric suau en $\mathbb R^3$.
\begin{defi}
    Anomenem \textbf{cinta de Möbius plana de raó d'aspecte $\lambda$} la varietat topològica amb frontera obtinguda amb la identificació d'un rectangle de $\mathbb R^2$
    $$M_\lambda := ([0,\lambda] \times [0,1])/\sim, \quad\quad (0,y)\sim(\lambda,1-y)$$
    dotada de la mètrica euclidiana heretada del rectangle.
\end{defi}

\begin{figure}[htbp]
    \centering
    \includegraphics[width=0.5\textwidth]{cinta}
    \caption{Cinta de Möbius plana de raó d'aspecte $\lambda$.}
    \label{fig:mobius_plana}
\end{figure}


\begin{defi}\label{def:cinta_mobius_paper}
    Una \textbf{cinta de Möbius de paper de raó d'aspecte $\lambda$} és una aplicació isomètrica $I:M_{\lambda}\to\mathbb R^3$ de classe $C^\infty$
\end{defi}
\begin{nota}
    Anomenarem $\Omega=I(M_\lambda)$ la imatge de $M_\lambda$ per $I$. Sovint anomenarem cinta de Möbius de paper a $\Omega$ en lloc de $I$. Quan $I$ sigui un encabiment, direm que $\Omega$ està \textbf{encabida} en $\mathbb R^3$.
\end{nota}

\begin{ex}
    Anomenem \textbf{cinta de Möbius de paper triangular} la cinta de Möbius de paper de raó d'aspecte $\lambda = \sqrt{3}$.
\end{ex}
\begin{figure}[htbp]
    \centering
    \includegraphics[width=0.3\textwidth]{cinta_triangular}
    \caption{Cinta de Möbius plana triangular.}
    \label{fig:mobius_triangular}
\end{figure}




A continuació presentem el teorema principal que demostrarem en aquest capítol. A més d'aquest, \citet{schwartz2024} demostra el \say{Teorema del límit triangular}, que enunciarem a les conclusions d'aquest capítol. 
\begin{teo}[\textbf{Principal}]\label{teo:Main Schwartz}
    Qualsevol cinta de Möbius de paper encabida en $\mathbb R^3$ té raó d'aspecte més gran que $\sqrt{3}$.
\end{teo}




\subsection{Preliminars a la demostració}
\begin{defi}
    Sigui $I:M_\lambda\to\Omega$ una cinta de Möbius de paper encabida. Un \textbf{plec} a $\Omega$ és un segment de recta $B'$ que talla a través de $\Omega$ i té els seus extrems a la frontera.
\end{defi}

\begin{defi}
    Sigui $B'$ un plec. Anomenem \textbf{pre-plec} a la preimatge $B=I^{-1}(B')$.
\end{defi}
Veurem més endavant que tota cinta de Möbius de paper encabida té una foliació per plecs, és a dir, que $\Omega$ es pot descompondre com una unió disjunta de plecs. A més, degut al fet que $I$ no incrementa distàncies, és fàcil veure que els pre-plecs també són segments de recta, i que $M_\lambda$ té una foliació per pre-plecs.
\begin{defi}
    Un \textbf{patró T en $\Omega$} (en anglès, \textit{T-pattern}) és un parell de plecs que es troben en rectes perpendiculars que intersequen en un punt. Diem que el patró T està \textbf{encabit} en $\Omega$ si els plecs no intersequen.
\end{defi}

El Teorema Principal es conseqüència directa dels dos lemes que enunciem a continuació. Només caldrà demostrar cada un d'ells.
\begin{lema}[\textbf{T}]\label{lema T}
    Tota cinta de Möbius de paper encabida té un patró T encabit.
\end{lema}

\begin{lema}[\textbf{G}]\label{lema G}
    Tota cinta de Möbius de paper encabida amb un patró T encabit té raó d'aspecte més gran que $\sqrt{3}$.
\end{lema}

\section{Demostració}
\subsection{Existència de la foliació per plecs}
Sigui $int(\Omega)$ l'interior de $\Omega$, i $U$ el subconjunt de $int(\Omega)$ format pels punts amb curvatura mitjana diferent de zero, tal com hem definit a l'equació \eqref{eq:curvatura_mitjana}.

Per demostrar l'existència de la foliació per plecs, utilitzarem el lema següent, que no demostrarem:
\begin{lema}\label{lema:2.2}
    Tot punt $p\in U$ pertany a un únic plec $\gamma$, i tot punt de $\gamma$ té curvatura mitjana diferent de zero, fins i tot als extrems.
\end{lema}
Podem parlar de curvatura mitjana en un punt de la frontera $p\in\partial \Omega$ prenent una extensió suau de $\Omega$ en un entorn de $p$.

\begin{teo}\label{teo:existencia_foliacio_plecs}
    $\Omega$ té una foliació per plecs.
\end{teo}
{
    \color{black!50!green}
    \textit{Prova.}
    Sigui $U^*$ la unió de tots els plecs que passen per punts de $U$. Observem que els punts de la frontera de cada plec pertanyen a $U^*$. Els punts de $U^*$ no intersequen, ja que altrament aquell punt tindria curvatura mitjana zero, contradient el lema \ref{lema:2.2}. Per tant, $U^*$ té una foliació per plecs. 

    Sigui $\tau'$ un component connex de $\Omega\setminus U^*$.
    Si $\tau'$ té interior buit, aleshores és el límit d'una successió de plecs de $U^*$, i per tant és també un plec en ser un segment de recta.
    Si $\tau'$ té punts al seu interior, aleshores tots ells tenen curvatura mitjana zero, de manera que pertanyen a un sol pla.
    Sigui $\tau = I^{-1}(\tau')$. $I|_{\tau}$ és un encabiment isomètric entre regions planes i $\tau$ és un trapezoide convex. Vegeu la figura \ref{fig:trapezoide}. Per tant, $I|_{\tau}$ és una isometria global, de manera que les imatges de segments de recta en $\tau$ són segments de recta. Podem obtenir una foliació per pre-plecs de $\tau$ de manera senzilla interpolant entre els dos pre-plecs de la seva frontera, i prendre la imatge per $I$ d'aquesta foliació per obtenir una foliació per plecs de $\tau'$.
    
    \begin{figure}[htbp]
        \centering
        \raisebox{0.2\height}{\includegraphics[width=0.5\textwidth]{verd2}}
        \quad
        \raisebox{2\height}{$\xleftarrow{\scalebox{1}{$I^{-1}$}}$}
        \quad
        \includegraphics[width=0.3\textwidth]{verd1}
        \caption{$\tau'$ en $\Omega$ i $\tau = I^{-1}(\tau')$ en $M_\lambda$.}
        \label{fig:trapezoide}
    \end{figure}

    Realitzant aquest procés per cada component connex de $\Omega\setminus U^*$, obtenim una foliació per plecs de $\Omega$.
    \qed
}
\subsection{Demostració del lema T}
\begin{defi}
    Sigui $I:M_\lambda\to\Omega$ una cinta de Möbius de paper encabida. 
    Anomenem \textbf{línia central de $M_\lambda$} al cercle $([0,\lambda]\times\set{1/2})/\sim$.
    Anomenem \textbf{línia central de $\Omega$} a la imatge de la línia central de $M_\lambda$ per $I$.
\end{defi}

\begin{defi}
    Sigui $u$ un plec. Anomenem \textbf{orientació de $u$} a cada un dels dos vectors unitaris paral·lels a $u$, $\pm \vec u$.
\end{defi}

\begin{prop}
    Tot pre-plec interseca la línia central de $M_\lambda$ en un únic punt.
\end{prop}
{
    \color{black!50!green}
    \textit{Prova.}
    Sigui $u$ un pre-plec, i $\ell(u)$ la seva longitud. Si $\ell(u)<\sqrt{1+\lambda^2}$, es pot desplaçar per una isometria per tal que no toqui les verticals del rectangle, de manera que clarament ha de tallar a la línia central en un únic punt.
    Suposem, per tant, que $\ell(u)\geq\sqrt{1+\lambda^2}$. Aleshores, $\ell(u)>\lambda$. Però la frontera $\partial \Omega$ és un cercle (topològic) de longitud $2\lambda$ que conté els extrems del plec $I(u)$. Així, $I(u)$ és un plec que divideix la frontera en dos arcs, i cada un d'aquests ha de tenir longitud més gran que $\ell(I(u))$. És a dir, $\ell(\partial \Omega) \ge  2\ell(I(u)) = 2\ell(u)>2\lambda$. Però això és una contradicció, ja que $\ell(\partial \Omega) = 2\lambda$.
    \qed
}
\begin{obs}
    Escollint una foliació per plecs $\beta$, el fet que $I$ és un encabiment implica que la proposició també és certa per plecs: tot plec interseca la línia central de $\Omega$ en un únic punt. Així, podem associar a cada plec de $\beta$ un únic valor de $\mathbb R/2\pi$
\end{obs}

Sigui $\Gamma$ l'espai de parelles $(x_0,x_1)\in(\mathbb R/2\pi)^2$ tals que $x_0\neq x_1$. Per la observació anterior, cada parella de plecs de $\beta$ diferents es correspon amb un únic element de $\Gamma$. $\Gamma$ és homeomorf a un cilindre. Vegeu la figura \ref{fig:cilindre}.

\begin{figure}[htbp]
    \centering
    \includegraphics[height=0.25\textwidth]{dibuix_cilindre_1}
    \raisebox{5\height}{$\quad\quad\Large\cong\quad$}
    \includegraphics[height=0.25\textwidth]{dibuix_cilindre_2}
    \caption{Cilindre de $\Gamma$.}
    \label{fig:cilindre}
\end{figure}

Podem compactificar $\Gamma$ afegint $\partial_+$ com el límit de les parelles en què $x_1$ va just abans que $x_0$ en el cilindre, i $\partial_-$ com el límit de les parelles en què $x_1$ va just després que $x_0$. Aquest nou espai, $\overline{\Gamma}$, és homeomorf a l'esfera $\mathbb S^2$. L'aplicació $\Sigma(x_0,x_1) = (x_0,x_1)$ es pot estendre a una aplicació contínua de l'esfera que bescanvia $\partial_+$ i $\partial_-$. Vegeu la figura \ref{fig:esfera}.

\begin{figure}[htbp]
    \centering
    \includegraphics[height=0.25\textwidth]{dibuix_esfera_1}
    \raisebox{5\height}{$\quad\quad\Large\cong\quad$}
    \includegraphics[height=0.25\textwidth]{dibuix_esfera_2}
    \caption{Esfera de $\overline{\Gamma}$.}
    \label{fig:esfera}
\end{figure}

Per tot parell $(x_0,x_1)\in{\Gamma}$, hi ha un únic camí positiu $\set{x_t}$ en l'ordre cíclic de $\mathbb R/2\pi$ que va de $x_0$ a $x_1$. Aquest camí té longitud propera a zero quan $(x_0,x_1)$ està a prop de $\partial_+$, i propera a $2\pi$ quan $(x_0,x_1)$ està a prop de $\partial_-$.

Escrivim $\vec u_0\rightsquigarrow\vec u_1$ per denotar que hi ha una orientació contínua de plecs $\set{u_t}$ amb extrems $u_0$ i $u_1$. Notem que, si $\vec u_0\rightsquigarrow\vec u_1$, aleshores $-\vec u_0\rightsquigarrow-\vec u_1$, ja que és un canvi d'orientació, i que $\vec u_1\rightsquigarrow-\vec u_0$, ja que ens trobem en una cinta de Möbius. 

Sigui $m_j$ el punt mig de $u_j$. Definim l'aplicació
\begin{align*}
    F=(g,h): {\Gamma}&\to\mathbb R^2,\\
    (x_0,x_1)&\mapsto(\vec u_0\cdot\vec u_1, (m_0- m_1)\cdot(\vec u_0\times\vec u_1))
\end{align*}
on $\vec u_0\rightsquigarrow\vec u_1$. Notem que la definició és independent de la orientació, ja que $\vec u_0\rightsquigarrow\vec u_1\iff-\vec u_0\rightsquigarrow-\vec u_1$, i que $F$ es pot estendre de manera contínua a $\mathbb S^2$ de la següent manera:
\begin{align*}
    \overline{F}=(\overline{g},\overline{h}): \overline{\Gamma}&\to\mathbb R^2,\\
    (x_0,x_1)&\mapsto\begin{cases}
        F(x_0,x_1) & \text{si } (x_0,x_1)\in{\Gamma},\\
        (\pm 1,0) & \text{si } (x_0,x_1)={\partial_\pm}.
    \end{cases}
\end{align*}

\begin{prop}
    $F$, tal com l'hem definit, és tal que $F\circ\Sigma = -F$.
\end{prop}
{
    \color{black!50!green}
    \textit{Prova.}
    Només cal veure que és cert per les funcions $g$ i $h$. Utilitzem que $\vec u_0\rightsquigarrow\vec u_1\iff-\vec u_0\rightsquigarrow-\vec u_1$.

    Com $g(x_0,x_1) = \vec u_0\cdot\vec u_1$, tenim que $g(x_1,x_0) = \vec u_1 \cdot (-\vec u_0) = -g(x_0,x_1)$.

    Com $h(x_0,x_1) = (m_0- m_1)\cdot(\vec u_0\times\vec u_1)$, tenim que $h(x_1,x_0) = (m_1- m_0)\cdot(\vec u_1\times(-\vec u_0)) = (m_1- m_0)\cdot(\vec u_0\times\vec u_1) = -h(x_0,x_1)$.
    \qed
}

A més, pels punts antipodals $\partial_+$ i $\partial_-$ tenim que $\overline{F}(\partial_+) = (1,0)$ i $\overline{F}(\partial_-) = (-1,0)$.

Com $\overline{F}$ és contínua una aplicació contínua sobre un domini homeomorf a una esfera tal que $\overline{F}\circ\Sigma = -\overline{F}$, pel teorema de Borsuk-Ulam, existeix algun punt del domini amb imatge nul·la. Aquest punt no pot ser $\partial_+$ ni $\partial_-$, ja que $\overline{F}(\partial_\pm) = (\pm1,0)$, per tant ha de pertànyer a $\Gamma$.

Siguin $(u_0,u_1)$ els plecs corresponents a $(x_0,x_1)\in F^{-1}(0,0)$.
Aleshores:
\begin{enumerate}
    \item $u_0$ i $u_1$ són disjunts, ja que són plecs diferents d'una mateixa foliació.
    \item $u_0$ i $u_1$ són ortogonals, ja que $g(x_0,x_1) = 0$.
    \item $u_0$ i $u_1$ són coplanars, ja que $h(x_0,x_1) = 0\implies$ $\vec u_0$, $\vec u_1$ i $m_0-m_1$ són tots ortogonals a $\vec u_0\times\vec u_1$.
\end{enumerate}

Tot plegat, implica que $u_0$ i $u_1$ formen un patró T encabit, com volíem demostrar.

\subsection{Demostració del lema G}
\begin{nota}
    Sigui $\ell$ la longitud d'arc, $\triangledown$ un triangle de base horitzontal i $\lor$ els costats no-horitzontals.
\end{nota}

\begin{defi}
    Si $\triangledown$ té base $\sqrt{1+t^2}$ i alçada $h\ge1$, aleshores $\ell(\lor)\ge\sqrt{5+t^2}$. La igualtat es compleix si i només si $h=1$ i $\triangledown$ és isòsceles.
\end{defi}

{
    \color{black!50!green}
    \textit{Prova.}
    Siguin $p_1$, $p_2$ i $q$ els vèrtexs de $\triangledown$, amb $p_1p_2$ la base. Sigui $p_2'$ el simètric de $p_2$ respecte de la recta horitzontal que passa per $q$. Aleshores,
    \begin{equation*}
        \ell(\lor) = \|p_1-q\| + \|p_2-q\| = \|p_1-q\| + \|p_2'-q\| \ge \|p_1-p_2'\| = \sqrt{1+t^2+4h^2}\ge\sqrt{5+t^2}.
    \end{equation*}
    La igualtat es compleix si i només si $h=1$ i $p_1p_2'$ és vertical, que és el cas de l'isòsceles.
    \qed
}

Sigui $I:M_\lambda\to\Omega$ una cinta de Möbius de paper encabida amb un patró T encabit.
\begin{nota}
    Per qualsevol subconjunt $S\subseteq M_\lambda$, anomenem $S'=I(S)$ la imatge de $S$ per $I$.
\end{nota}

Rotem $\Omega$ tal que un plec sigui horitzontal, $T'$, i l'altre estigui per sota en el mateix pla, $B'$. Observant $\Omega$ i $M_\lambda$, verifiquem que tenen la forma de la figura \ref{fig:mobius_1}, on la representació de $\Omega$ és una simplificació del que es veuria si projectéssim $\Omega$ en el pla format per $T'$ i $B'$.

\begin{figure}[htbp]
    \centering
    \includegraphics[height=0.33\textwidth]{dibuix_mobius_1}
    \quad
    \raisebox{4\height}{$\xrightarrow{\scalebox{1.5}{$I$}}$}
    \quad
    \includegraphics[height=0.33\textwidth]{dibuix_mobius_2}
    \caption{Cinta de Möbius plana tallada per $T$ (dreta) i cinta de Möbius de paper encabida amb un patró T encabit (esquerra).}
    \label{fig:mobius_1}
\end{figure}

Ara observem que 
\begin{equation*}
    \begin{cases}
        \text{base} &= \ell(T') = \ell(T) = \sqrt{1+t^2},\\
        \text{alçada} &> \ell(B') = \ell(B) = \sqrt{1+b^2}\ge1.
    \end{cases}
\end{equation*}

Per tant, es verifiquen
\begin{equation*}
    \begin{cases}
        \ell(H) + \ell(D) = 2\lambda\\
        \ell(D)-2t=\ell(H)\\
        \sqrt{1+t^2}=\ell(T')\le\ell(H')=\ell(H)\\
        \sqrt{5+t^2}<\ell(\lor)\le\ell(D')=\ell(D)
    \end{cases}
\end{equation*}

Definim les corbes
\begin{equation*}
    \begin{cases}
        \alpha(t) := \sqrt{1+t^2} + \sqrt{5+t^2}<\ell(H) + \ell(D) = 2\lambda\\
        \beta(t) := 2\sqrt{5+t^2}-2t < 2\ell(D)-2t = \ell(D) + \ell(H) = 2\lambda
    \end{cases}
\end{equation*}

Per tant, $2\lambda > \max(\alpha(t),\beta(t))$. Posant $t_0 = 1/\sqrt{3}$, tenim que $\alpha(t_0) = \beta(t_0) = 2\sqrt3$. 
Observem que $\alpha$ és creixent en $(0,\infty)$. Per tant, $\alpha(t)>2\sqrt3$ per a tot $t>t_0$. En canvi, $\beta$ és decreixent en $t\in\mathbb R$. Per tant, $\beta(t)>2\sqrt3$ per a tot $t<t_0$. Tot plegat, obtenim que $\lambda>\sqrt3$, com volíem demostrar.

\section{Conclusions}
Amb això hem vist i demostrat una nova restricció sobre els encabiments isomètrics d'una varietat amb vora tan familiar com és la cinta de Möbius. De fet, l'article \cite{schwartz2024} va un pas més enllà, demostrant el següent teorema sobre successions de cintes de Möbius de paper encabides amb raons d'aspecte convergint a $\sqrt{3}$:

\begin{teo}[\textbf{Límit Triangular}]\label{teo:Límit triangular}
    Sigui $I_n:M_{\lambda_n}\to\mathbb R^3$ una successió de cintes de Möbius de paper encabides, tals que $\lambda_n\to\sqrt{3}$. Aleshores, $I_n$ convergeix uniformement a una cinta de Möbius de paper triangular, llevat d'isometria.
\end{teo}

Ara bé, hi ha una manera de prendre un rectangle de paper de raó d'aspecte $\lambda = 1$, i doblegar-lo de tal manera que el resultat sigui \textit{gairebé} idèntic a una cinta de Möbius de paper encabida. Per aconseguir això, només cal prendre el tros de paper i doblegar-lo un nombre parell de vegades, de tal manera que quedi dividit en un nombre senar de seccions iguals. Si aquestes seccions es dobleguen de manera alternant, com un acordió, podem enganxar els dos extrems i aconseguir un \textbf{acordió de Möbius}. De fet, amb aquest mètode podem prendre un rectangle de paper de qualsevol raó d'aspecte i tornar-lo en un acordió de Möbius. 

\begin{figure}[htbp]
    \centering
    \includegraphics[width=0.25\textwidth]{acordió.pdf}
    \caption{Acordió de Möbius.}
    \label{fig:acordio_mobius}
\end{figure}

Què té d'especial l'acordió de Möbius que impedeix que serveixi com a contraexemple del teorema Principal? Aquest objecte no és una cinta de Möbius de paper encabida, tal com l'hem definit a la definició \ref{def:cinta_mobius_paper}, perquè els dos extrems no poden encaixar correctament doblegant el paper de manera suau. Els punts per on dobleguem, dit d'altra manera, no són regulars, de manera que l'encabiment no és $C^\infty$. En concret, hem necessitat regularitat com a mínim $C^2$ per poder definir la curvatura mitjana en la demostració del teorema \ref{teo:existencia_foliacio_plecs}. Si l'encabiment no és prou regular, no existeix la foliació per plecs necessària per demostrar el teorema Principal. 

L'acordió de Möbius deixa clar que, relaxant les condicions de regularitat, aconseguim encabiments isomètrics de varietats que no n'admetrien altrament. Aquest fet destacable és el que podem generalitzar en el capítol que segueix, on estudiarem el Teorema de Nash d'immersions i encabiments isomètrics $C^1$.

\newpage