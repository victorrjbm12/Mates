\chapter{Cinta de Moebius de paper òptima}
{\color{blue}La idea d'aquest capítol és la següent:
\begin{itemize}
    \item Introduir que estem parlant d'això per mostrar d'una altra manera la rigidesa dels encabiments suaus.
    \item Explicar el problema de la cinta de Moebius de paper òptima.
    \item Enunciar i demostrar el teorema principal.
    \item Enunciar i demostrar el teorema del límit triangular.
    \item Explicar l'acordió de Möbius, donant peu al següent capítol.
\end{itemize}
}
\section{Introducció al capítol}
{\color{blue}En aquesta secció expliquem el paper de Richard Evan Schwartz, en què demostra que la cinta de Moebius de paper òptima ha de tenir relació d'aspecte més gran que $\sqrt{3}$, i que qualsevol successió de cintes de Moebius amb relació d'aspecte convergint a $\sqrt{3}$ convergeix a la cinta de Moebius triangular.}
\subsection{Introducció al problema}
\begin{defi}
    Una {\normalfont cinta de Moebius de paper de relació d'aspecte $\lambda$} és una aplicació infinitament diferenciable (suau) $I:M_{\lambda}\to\mathbb R^3$, on $M_\lambda$ és la cinta de Moebius plana obtinguda amb la següent identificació d'un rectangle
    $$M_\lambda = ([0,\lambda] \times [0,1])/\sim, \quad\quad (0,y)\sim(\lambda,1-y)$$
Una {\normalfont aplicació isomètrica} és una aplicació que preserva longituds d'arc. L'aplicació és una {\normalfont encabiment} si és injectiva, i una {\normalfont immersió} en general {\color{blue} Interessant veure si aquesta notació la seguim utilitzant o si se l'ha inventat}. Sigui $\Omega=I(M_\lambda)$. Diem que $\Omega$ està {\normalfont incrustada} si $I$ és una encabiment. 
\end{defi}
\begin{ex}
    Anomenem \textit{cinta de Moebius de paper triangular} la cinta de Moebius de paper de relació d'aspecte $\lambda = \sqrt{3}$.
\end{ex}
{\color{blue} el shwartz menciona aquí uns papers que expliquem més sobre la banda de moebius, si volem escriure més sobre el tema els podem utilitzar.}

\begin{teo}[Principal]\label{teo:Main Schwartz}
    Una cinta de Moebius de paper suau incrustada en $\mathbb R^3$ té relació d'aspecte més gran que $\sqrt{3}$.
\end{teo}

\begin{teo}[Límit Triangular]\label{teo:Límit triangular}
    Sigui $I_n:M_{\lambda_n}\to\mathbb R^3$ una successió de cintes de Moebius de paper suaus incrustades, tals que $\lambda_n\to\sqrt{3}$. Aleshores, $I_n$ convergeix uniformement a una cinta de Moebius de paper triangular, llevat d'isometria.
\end{teo}
\subsection{Definicions}
\begin{defi}
    Sigui $I:M_\lambda\to\Omega$ una cinta de Moebius de paper incrustada. Un {\normalfont plec} {\color{blue} "doblec" o alguna cosa així?} a $\Omega$ és un segment de recta $B'$ que talla a través de $\Omega$ i té els seus extrems a la frontera.
\end{defi}
Veurem més endavant que tota cinta de Moebius de paper incrustada té una foliació per plecs.
\begin{defi}
    Sigui $B'$ un plec. Anomenem {\normalfont pre-plec} a la preimatge $B=I^{-1}(B')$.
\end{defi}
Degut al fet que $I$ no incrementa distàncies, és fàcil veure que els pre-plecs també són segments de recta, i que $M_\lambda$ té una foliació per pre-plecs.

\subsection{Lemes addicionals}
Per demostrar el teorema principal, es necessiten aquests dos lemes.
\begin{lema}[T]\label{lema T}
    Una cinta de Moebius de paper incrustada suau té un  

\end{lema}
\newpage