\documentclass{beamer}

%------------------------------------------------------------
% Paquets
%------------------------------------------------------------
\usepackage[utf8]{inputenc}
\usepackage[catalan]{babel}
\usepackage{amsmath, amssymb}
\usepackage{graphicx}

%------------------------------------------------------------
% Tema i colors
%------------------------------------------------------------
\usetheme{Madrid}
\usecolortheme{default}

%------------------------------------------------------------
% Informació del títol
%------------------------------------------------------------
\title[Encabiments Isomètrics $C^1$ i $C^\infty$]{Encabiments Isomètrics $C^1$ i $C^\infty$ en $\mathbb{R}^n$}
\author{Víctor Rubio Jiménez}
\institute{Facultat de Matemàtiques i Informàtica \\ Universitat de Barcelona}
\date{Defensa del Treball Final de Grau - 27 de juny de 2025}
\titlegraphic{\includegraphics[height=1.5cm]{matematiquesinformatica-pos-rgb.png}}

%------------------------------------------------------------
% Inici del document
%------------------------------------------------------------
\begin{document}

% --- Diapositiva de Títol ---
\begin{frame}
  \titlepage
\end{frame}

% --- Diapositiva d'Índex ---
\begin{frame}
  \frametitle{Índex de la presentació}
  \tableofcontents
\end{frame}

%------------------------------------------------------------
\section{Introducció: El Problema dels Encabiments Isomètrics}
%------------------------------------------------------------

\begin{frame}
  \frametitle{Varietats riemannianes}
  \begin{itemize}
    \item Una \textbf{varietat topològica de dimensió} $n$ és un espai topològic Hausdorff, 2-numerable i localment homeomorf a $\mathbb{R}^n$.
    \item Una \textbf{carta coordenada} és un parell $(U, \varphi)$ on $U$ és un obert de $M$ i $\varphi: U \to \mathbb{R}^n$ és un homeomorfisme de $U$ amb la seva imatge.
    \item Una \textbf{varietat diferenciable $C^k$} és una varietat topològica amb una estructura diferenciable, és a dir, un conjunt maximal de cartes coordenades $\{(U_\alpha, \varphi_\alpha)\}$ que recobreixen la varietat i tal que $\varphi_\alpha \circ \varphi_\beta^{-1}$ és $k$ vegades diferenciable per a tot $\alpha, \beta$ tals que $U_\alpha \cap U_\beta \neq \emptyset$.
    \item Una \textbf{varietat riemanniana} és una varietat diferenciable amb una mètrica riemanniana, és a dir, una aplicació $g$ que assigna a cada punt $p \in M$ un producte intern $g_p: T_pM \times T_pM \to \mathbb{R}$.
  \end{itemize}
\end{frame}

\begin{frame}
    \frametitle{Encabiments de varietats diferenciables}
    \begin{itemize}
      \item Una \textbf{varietat topològica de dimensió} $n$ és un espai topològic Hausdorff, 2-numerable i localment homeomorf a $\mathbb{R}^n$.
      \item Una \textbf{carta coordenada} és un parell $(U, \varphi)$ on $U$ és un obert de $M$ i $\varphi: U \to \mathbb{R}^n$ és un homeomorfisme de $U$ amb la seva imatge.
      \item Una \textbf{varietat diferenciable $C^k$} és una varietat topològica amb una estructura diferenciable, és a dir, un conjunt maximal de cartes coordenades $\{(U_\alpha, \varphi_\alpha)\}$ que recobreixen la varietat i tal que $\varphi_\alpha \circ \varphi_\beta^{-1}$ és $k$ vegades diferenciable per a tot $\alpha, \beta$ tals que $U_\alpha \cap U_\beta \neq \emptyset$.
      \item Una \textbf{varietat riemanniana} és una varietat diferenciable amb una mètrica riemanniana, és a dir, una aplicació $g$ que assigna a cada punt $p \in M$ un producte intern $g_p: T_pM \times T_pM \to \mathbb{R}$.
    \end{itemize}
  \end{frame}

\begin{frame}
  \frametitle{Què és un encabiment isomètric?}
  
  Una varietat riemanniana $(M, g)$ té una geometria \textbf{intrínseca}, definida per la seva mètrica $g$. 
  
  \vspace{1em}
  
  Si l'encabim en un espai euclidià $\mathbb{R}^n$ amb una aplicació $f: M \to \mathbb{R}^n$, la mètrica euclidiana indueix una geometria \textbf{extrínseca} sobre la imatge $f(M)$. 
  
  \vspace{1em}
  
  \begin{block}{Pregunta fonamental}
    Quan podem trobar un encabiment $f$ tal que la geometria intrínseca i l'extrínseca coincideixin? És a dir, un \textbf{encabiment isomètric}. 
  \end{block}
  
  \vspace{1em}
  
  La resposta depèn crucialment de la \textbf{regularitat} (suavitat) de l'aplicació $f$. 
\end{frame}

%------------------------------------------------------------
\section{La Rigidesa dels Encabiments $C^\infty$}
%------------------------------------------------------------

\begin{frame}
  \frametitle{La Rigidesa dels Encabiments $C^\infty$}
  
  Els encabiments suaus ($C^2$ o superior) són molt \textbf{rígids}. Les seves propietats geomètriques, com la curvatura, estan molt restringides. 
  
  \begin{block}{Teorema 1: El Tor Pla (Resultat clàssic)}
    \textbf{No existeix} cap encabiment isomètric de classe $C^\infty$ (ni tan sols $C^2$) d'un tor pla en $\mathbb{R}^3$. 
  \end{block}
  
  \begin{itemize}
    \item \textbf{Motiu}: El \textit{Teorema Egregi} de Gauss afirma que la curvatura gaussiana és un invariant isomètric. 
    \item Un tor pla té curvatura $K=0$ a tot arreu. 
    \item Però, tota superfície compacta en $\mathbb{R}^3$ ha de tenir almenys un punt de curvatura positiva (un punt el·líptic). 
  \end{itemize}
  
  Aquesta contradicció demostra la impossibilitat. 
\end{frame}

\begin{frame}
  \frametitle{Un altre exemple de rigidesa: La Cinta de Möbius}
  
  Fins i tot per a varietats que sí que es poden encabir, hi ha restriccions fortes.
  
  \begin{block}{Teorema 2: La Cinta de Möbius "de paper" (Schwartz, 2024)}
    Una cinta de Möbius rectangular, per poder ser encabida isomètricament i de forma $C^\infty$ en $\mathbb{R}^3$, ha de tenir una raó d'aspecte $\lambda$ (llargada/amplada) estrictament més gran que $\sqrt{3}$. 
    $$ \lambda > \sqrt{3} $$
  \end{block}
  
  \begin{center}
    %   \includegraphics[width=0.6\textwidth]{mobius_aspect.png} % Hauries de crear aquesta imatge a partir de la Figura 3.1 i 3.2
  \end{center}
  
  Això demostra que la geometria intrínseca ($g$) imposa condicions molt restrictives sobre la forma extrínseca ($f(M)$) en el món $C^\infty$. 
  
\end{frame}


%------------------------------------------------------------
\section{La Flexibilitat dels Encabiments $C^1$}
%------------------------------------------------------------

\begin{frame}
  \frametitle{La Flexibilitat dels Encabiments $C^1$}
  
  Què passa si relaxem la condició de regularitat de $C^\infty$ a $C^1$?
  \begin{itemize}
      \item La mètrica induïda continua estant ben definida.
      \item Però la curvatura (que depèn de les segones derivades) \textbf{no ho està}. 
  \end{itemize}

  \vspace{1em}
  
  \begin{alertblock}{El Teorema de Nash-Kuiper (1954-55)}
    Qualsevol encabiment $C^\infty$ \textbf{curt} d'una varietat riemanniana $n$-dimensional en $\mathbb{R}^{n+k}$ (amb $k \ge 1$) pot ser aproximat arbitràriament per un \textbf{encabiment isomètric de classe $C^1$}. 
  \end{alertblock}
  
  Això implica que les restriccions de curvatura desapareixen i s'obre la porta a resultats sorprenents.
\end{frame}

\begin{frame}
  \frametitle{Com és possible? La idea de la demostració}
  
  La demostració de Nash és \textbf{constructiva i iterativa}. 
  
  \begin{enumerate}
    \item \textbf{Punt de partida}: Un encabiment $C^\infty$ que "escurça" totes les distàncies (un encabiment curt). 
    
    \item \textbf{Procés iteratiu}: S'afegeixen pertorbacions a l'encabiment en cada etapa.
    
    \item \textbf{Les pertorbacions}: Són "arrugues" o "corrugacions" d'alta freqüència i baixa amplitud.  Aquestes corrugacions "estiren" la superfície el suficient per corregir el defecte isomètric.
    
    \item \textbf{Convergència}: El procés convergeix a una aplicació que és:
    \begin{itemize}
        \item \textbf{Isomètrica}: Les distàncies són les correctes.
        \item \textbf{De classe $C^1$}: Les derivades primeres convergeixen.
        \item \textbf{No $C^2$}: Les derivades segones (i per tant la curvatura) no convergeixen. 
    \end{itemize}
  \end{enumerate}
  
  Aquesta tècnica va donar lloc a la \textbf{integració convexa}. 
\end{frame}

%------------------------------------------------------------
\section{Aplicació: El Tor Pla en $\mathbb{R}^3$}
%------------------------------------------------------------

\begin{frame}
  \frametitle{El Retorn del Tor Pla}
  
  El teorema de Nash-Kuiper suggereix que la impossibilitat d'encabir el tor pla en $\mathbb{R}^3$ podria ser un artefacte de la rigidesa $C^\infty$.
  
  \vspace{1em}
  
  \begin{block}{Construcció $C^1$ (Borrelli et al., 2013)}
    Utilitzant mètodes moderns d'integració convexa, es pot construir explícitament un \textbf{encabiment isomètric i de classe $C^1$} d'un tor pla en $\mathbb{R}^3$. 
  \end{block}
  
  \begin{center}
    % \includegraphics[width=0.8\textwidth]{torus_c1.png}
    \tiny{Font: Borrelli et al. (2013). }
  \end{center}
  
  Aquest objecte, tot i ser matemàticament un tor pla, té una estructura fractal i extremadament "arrugada", completament contraintuïtiva. 
\end{frame}

%------------------------------------------------------------
\section{Conclusions}
%------------------------------------------------------------

\begin{frame}
  \frametitle{Conclusions}
  
  El treball explora la dicotomia entre la rigidesa i la flexibilitat en els encabiments isomètrics.
  
  \begin{itemize}
    \item \textbf{Regularitat $C^\infty \implies$ Rigidesa}: 
    \begin{itemize}
        \item Fortes restriccions geomètriques basades en la curvatura.
        \item Impossibilitat d'encabir el tor pla en $\mathbb{R}^3$. 
        \item Condicions geomètriques per a la cinta de Möbius ($\lambda > \sqrt{3}$). 
    \end{itemize}
    
    \vspace{1em}
    
    \item \textbf{Regularitat $C^1 \implies$ Flexibilitat}: 
    \begin{itemize}
        \item Les restriccions de curvatura desapareixen.
        \item Permet construccions "impossibles" i contraintuïtives, com l'encabiment $C^1$ del tor pla a $\mathbb{R}^3$. 
        \item El teorema de Nash-Kuiper formalitza aquesta flexibilitat. 
    \end{itemize}
  \end{itemize}
  
  La pèrdua d'un grau de regularitat canvia radicalment el panorama geomètric.
\end{frame}

% --- Diapositiva Final ---
\begin{frame}
  \begin{center}
    \Huge\bfseries
    Gràcies per la vostra atenció.
    \vspace{2em}
    
    Preguntes?
  \end{center}
\end{frame}

\end{document}